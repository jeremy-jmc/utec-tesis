%% --------------------------------------------------------------------
%% Template UTEC Tesis
%% --------------------------------------------------------------------
% Este es el archivo que se debe compilar.
% 
% Este template ha sido modificado y actualizado por Eduardo Castro y Roosevelt Ubaldo en base a lo trabajado por Víctor Murray, Oscar Ramos y Juan Carlos Barbaran.
%
% Última actualización: Mayo, 2022

\documentclass[a4paper,12pt,oneside]{tesisutec}


\selectlanguage{spanish}
%% Paquetes
\usepackage[utf8]{inputenc}
\usepackage[square, numbers, comma, sort&compress]{natbib}
% Libreria de idioma
\usepackage[spanish]{babel}
% Libreria para posicionamiento
\usepackage{float}
% Librerias para insertar codigos
\usepackage[spanish,onelanguage,ruled,vlined]{algorithm2e}
\usepackage{verbatim} 
% Librería para hipervínculos
\usepackage{hyperref}
 % Librería necesaria para arreglar el orden de referencias en overleaf.com
\usepackage{notoccite}
\usepackage{amsmath}
% \usepackage{multirow}


% Incluir acá los paquetes adicionales que deseas. 
% Ubicación de los imágenes.
\graphicspath{images/}
\usepackage{svg}
\makeatletter
% Reinsert missing \algbackskip
\def\algbackskip{\hskip-\ALG@thistlm}
\makeatother

\hypersetup{urlcolor=blue, colorlinks=true}

\begin{document}

\frontmatter

\department{CIENCIA DE LA COMPUTACIÓN}


\title{CrimeLens Chat: An LLM-Powered Natural-Language and Visual Analytics Interface for Local Crime Exploration}
% Titulos alternativos:

% “CrimeGPT: Conversational Spatial‑Temporal Crime Analysis with Interactive Visualizations”



\author{Marcos Daniel Ayala Pineda \\ Jeremy Jeffrey Matos Cangalaya} % Es obligatorio agregar ORCID del alumno
\supervisor{Cristian Lopez \\ Germain Garcia} % Es obligatorio agregar ORCID del asesor

\date{2025}
\degree{Licenciado }
\major{Ciencia de la computación}
\maketitle

\setstretch{1.5}


%\input{encabezados/dedicatoria}
%\input{encabezados/agradecimientos}

\tableofcontents



\addtocontents{toc}{\vspace{1.5em}}

%% ============================================================================
\mainmatter
\pagestyle{fancy}

% \input{secciones/resumen}
% \customchapter{ABSTRACT} 

% \begin{center}
% \large \vspace{-1.5cm} \textbf{A CONVERSATIONAL LLM-POWERED SYSTEM FOR SPATIO-TEMPORAL CRIME DATA ANALYSIS}
% \end{center}

Large Language Models (LLMs) have demonstrated exceptional versatility across diverse domains, yet their application in crime analytics remains underexplored due to a lack of domain-specific datasets and specialized methodologies. To address this gap, we introduce ChinaCrimesQACode, a novel dataset designed to capture the intricacies of spatio-temporal crime analysis, including detailed crime incidents, street networks, and administrative boundaries with location-specific queries. Leveraging ChinaCrimesQACode, we focus on code generation from natural language questions, enabling LLMs to produce high-quality, executable Python code for geospatial crime data analysis from structured conversational inputs. Fine-tuned model and baseline models were evaluated using semantic equivalence metrics and LLM-as-a-judge. Our results demonstrate substantial improvements in generating contextually accurate analytical code, highlighting the transformative potential of tailored datasets and fine-tuning methodologies in optimizing crime intelligence workflows. This work highlights the potential of LLMs in crime analytics workflows and the essential role of domain-specific datasets in tailoring them to specialized analytical challenges.

\noindent \textbf{Keywords:}\\
\noindent Large Language Models; Crime Analytics; Geospatial Analysis; Code Generation; Conversational AI 
\chapter{INTRODUCTION} 
% ¿Cuál es el contexto del tema? Contextualice al lector, inicia con un párrafo que le genere interés por el tema.
% ¿Cuál es el problema?
% ¿Qué han hecho otros para resolver el problema?
% ¿Qué propone el artículo?
% ¿En qué se diferencia tu trabajo del resto? (Cuál es el aporte)


% \section{Presentación del tema de investigación}   % Antecedentes, problematica y justificacion
\section{Introduction to the Research Topic} % TODO: Focus on social impact and problematic

Urban crime persists as a critical social challenge, with the \emph{Global Organized Crime Index 2023} reporting that 83\,\% of the world’s population now lives in countries exhibiting high levels of criminality \cite{GlobalCrimeIndex2023}. % :contentReference[oaicite:0]{index=0} 
Such criminality is rarely uniform within a city; instead, incidents concentrate at specific micro-places and times, creating dynamic “hot” and “cool” spots that evolve over days, weeks, and seasons \cite{Garcia2022CriPAV}.
Yet many jurisdictions still struggle with incomplete or delayed reporting, obscuring timely understanding of local risk patterns \cite{NSSFCrimeReporting2023}. Consequently, both authorities and citizens require tools that transform raw spatio-temporal crime data into actionable insights capable of guiding preventive decisions in real time.


A wide range of visual-analytics systems has been proposed to meet this need \cite{Garcia2022CriPAV, Salah2022BigCDVis, Silva2017CrimeVisAI, Garcia2020MiranteAV, Garcia2021CrimAnalyzer}. Despite their analytical power, these dashboards often overwhelm non-experts, demand familiarity with multiple interaction widgets, and provide limited narrative guidance, hindering adoption outside specialist circles. In response to this, Natural-language interfaces (NLIs) have emerged to lower these usability barriers \cite{Setlur2016Eviza, Narechania2021NL4DV, Luo2022NL2Vis, Liu2021ADVISor, Sah2024GeneratingAnalyticsDataVizLLMs}. However, most NLI pipelines rely on template grammars or traditional semantic-parsing models and thus still struggle with ambiguity, underspecification, and multi-step analytical reasoning.
 

Recent advances in large language models (LLMs) suggest a transformative alternative. LLM-powered assistants can interpret free-form questions, decompose them into analytical sub-tasks, and generate both textual explanations and visualization specifications \cite{Zhang2024NLITabular}. Yet, to date, no end-to-end platform combines the conversational flexibility of LLMs with geovisual analytics for crime data, leaving a usability and situational-awareness gap precisely where citizens and frontline officers need rapid, contextual insights.

Adopting an LLM-driven chat interface for spatio-temporal crime data directly addresses this gap. First, natural dialogue eliminates steep learning curves by allowing users to express information needs e.g., “Which streets saw the largest increase in robberies last weekend in this city?” without mastering dashboard controls.  Second, the model can highlight relevant data used to answer questions, enabling users to validate insights and build trust in the system. By unifying conversational AI with visual responses, the proposed research seeks to democratize access to crime intelligence, bolster data-driven policing, and ultimately enhance public safety for both residents and visitors.


% Data visualization is a powerful tool for data exploration and insight communication \cite{Liu2024NLDriven}.

% Robust and complete natural language interfaces is difficult to achieve, due to the language is often ambiguous and underspecified \cite{Setlur2019InferencingVisualAnalysis}, requiring extensive parsing and complex reasoning.

% Question answering (QA) is a research area that combines research from various fields, such as Information Retrieval (IR), Information Extraction (IE), and Natural Language Processing (NLP) \cite{Liu2021ADVISor}.

% The majority of the existing research address tabular data that is not necessarily georeferenced, calling into question its adaptability to space-time scenarios such as crime phenomena

% DL-based crime prediction encounters significant obstacles, such as concerns about fairness, accountability, transparency, interpretability, security, and privacy \cite{Ersoz2025CrimePredictionXAISurvey}.

% \section{Descripción de la situación problemática} % Planteamiento del problema
\section{Description of the Problematic Situation}

% * Paragraph 1 – General Context
In many large cities, concerns about crime have become increasingly common, especially in areas frequently visited by tourists \cite{SopikoTevdoradze2024CrimeTourism}. Popular destinations are often associated with considerable levels of criminality, which can discourage visitors from exploring freely \cite{Machado2012CrimeRJ}. However, the impact of urban crime extends beyond tourists. Local residents, while more familiar with their surroundings, typically rely on personal experience, local news, and community word-of-mouth to decide which streets or neighborhoods to avoid. This informal awareness helps them navigate perceived danger zones, but it does not fully shield them from the risks that exist in their own cities.

% Many popular tourist destinations are often associated with considerable crime rates, which can deter visitors and affect their willingness to explore these areas. Besides, big cities like New York and Chicago 
% For instance, cities like New York, Los Angeles, and Chicago have been known to experience significant crime rates, particularly in certain neighborhoods. This can lead to a perception of danger among tourists and residents alike, ultimately affecting their willingness to explore these areas.  negatively impact local economies.

% * Paragraph 2 – Impact and Limitations of Current Systems
In response to these concerns, several visualization tools \cite{Garcia2022CriPAV, Salah2022BigCDVis, Silva2017CrimeVisAI, Garcia2020MiranteAV, Garcia2021CrimAnalyzer} have emerged to increase awareness, validate hypotheses, and support safer decision-making. However, these solutions are often difficult to use for non-expert users. They usually provide sophisticated analysis using robust feature engineering processes, restricting their accessibility to a broader audience. As a result, both citizens and authorities are left with insufficient support to anticipate risks or act proactively.

% * Paragraph 3 – Analytical Challenges
A key challenge in crime prevention is the lack of accessible tools that help both citizens and authorities interpret complex crime data. While large volumes of data are publicly available \cite{Zhang2025CrimeDatasetChina, NYCDataset, ChicagoDataset}, they're often difficult to analyze without technical skills or time \cite{Zengli2020CrimePatterns}. Crime patterns constantly shift across time and space. Some areas may be riskier at night, others during weekends or certain seasons. Without tools that clearly communicate these dynamics in real time, people struggle to make safe decisions, and authorities face obstacles in allocating resources effectively.

% * Paragraph 4 – Gaps in SOTA Methods
Although recent tools have improved the visual representation of crime data, they still present critical limitations in usability and applicability. Many platforms depend heavily on user expertise and offer mostly static or pre-defined visualizations that demand manual interpretation \cite{Garcia2022CriPAV}. This limits their value in dynamic, real-world scenarios where fast decisions are required. Without systems that deliver clear, real-time insights tailored to a user's context, both citizens and authorities remain underserved in their efforts to understand and respond to urban crime.

% TODO: Users may lack the knowledge or skills to interact with the visualization effectively. Without proper guidance or intuitive interaction design, users may struggle to fully leverage the interactive capabilities of the visualization. \cite{Liu2024NLDriven}

% \section{Formulación del problema} % Planteamiento del problema
% \section{Problem Statement} % Statement of the problem


% \section{Justificación}    % Antecedentes, problematica y justificacion
\section{Justification}

% TODO: ! Paragraph 1: Context and Urgency


% ! Paragraph 2: Enabling Technology
Large language models (LLMs) offer a human-centred interface for interacting with complex data \cite{Yang2024HumanAIInteraction, Pappula2023LLMsFC}. Unlike, web dashboards, GIS (Geographic Information System) software or specialized tools, LLM-powered chatbot lets users ask questions like ``Which streets saw the largest surge in robberies in the last month?'' and instantly receive concise explanations. In addition, open-source models alleviate many of the privacy concerns \cite{Temsah2025DeepSeek, Ersoz2025CrimePredictionXAISurvey} often associated with proprietary systems and can be more cost-effective \cite{Liu2024NLDriven}. 
% Many research like \cite{Liu2024NLDriven} uses LLMs to assist in dataset constructions and training small LLMs, because it can be more cost-effective and data-privacy friendly.

% LLMs have significantly enhanced chatbots and virtual assistants, enabling more contextually aware and human-like interactions across various domains 
% With the advent of LLMs, there has been a growing in the interaction via chat to several tasks, like web-browsing, text summarization, question-answering, code generation, among others.

% ! Paragraph 3: Rationale for an LLM Layer 
Adopting an LLM-based conversational layer directly addresses the two principal weaknesses of current crime-analytics platforms: usability and applicability. First, a chat interface removes the steep learning curves of existing tools. Second, by supporting rapid, iterative exploration of temporal windows, geographic areas and crime types, it delivers immediate insights when users need them.

% ! Paragraph 4: Anticipated Impact & Wider Significance 
In this study, we aim to enhance crime-prevention efforts and raise awareness about the risks associated with routes used by tourists and residents by developing an intuitive chat interface. This interface will allow users to interact with crime data, presenting both textual summaries and dynamic visual feedback, to facilitate informed decision-making. Additionally, the work establishes a foundation for future investigations into the application of LLMs to more sophisticated crime-analysis tasks. % , such as prediction, topological analysis, and clustering | leveraging in the CoT process

% \section{Objetivos de investigación}   % Objetivos
% \section{Scope and Limitations / Constraints} % Scope and expected benefits
% \section{Alcance y limitaciones / restricciones} % Alcance y beneficios esperados
\section{Research Objectives} % Objectives

\subsection{General Objectives}
To develop an chat-based system that leverages LLMs to enhance crime prevention efforts by enabling users to interact with spatio-temporal crime data through natural language queries, thereby improving decision-making for both citizens and authorities.

\subsection{Specific Objectives}
\begin{itemize}
    \item Design and implement a chat-based interface that allows users to query crime data and receive both textual and visual feedback.
    \item Create a question-answering dataset specifically tailored for crime data.
    \item Address privacy and security concerns by focusing on the use of open-source LLMs as alternative solutions to proprietary models \cite{Temsah2025DeepSeek, Ersoz2025CrimePredictionXAISurvey}. 
    \item Implement and agent-based system that utilizes LLMs to process and analyze spatio-temporal crime data, providing insights and recommendations to users.
    % \item Lay the groundwork for future research on advanced LLM-based analyses, such as prediction, topological analysis, and clustering.
\end{itemize}

% TODO: add contributions section
% \section{Contributions}
\chapter{THEORETICAL FRAMEWORK}


\section{Spatio-Temporal Data}

The spatio-temporal data is a type of data that contains information about the spatial and temporal dimensions of an event or phenomenon. 
This dual indexing enables the representation of complex relationships between spatial and temporal elements, allowing for a more comprehensive understanding of the data.

The spatio-temporal data can be formalized as a tuple $ST = \{ (s_i, t_j, X_{ij} | s_i \in S , t_j \in T , X_{ij} \in \mathbb{R}^d )\}$, where $S$ is the spatial domain, $T$ is the temporal domain, and $X_{ij}$ signifies the observed attributes at location $s_i$ and time $t_j$.


%\section{Textual Graphs}

%\subsection{Knowledge Graphs}

%\subsection{Spatial Graphs}

%\subsection{Temporal Graphs}

%\subsection{Spatial-Temporal Graphs}




\section{Large Language Models (LLMs)}
% BERT

Large Language Models (LLMs) are built upon the transformer architecture, originally introduced by \cite{vaswani2023attentionneed}. This architecture revolutionized natural language processing (NLP) by replacing recurrent neural networks (RNNs) with self-attention mechanisms, enabling models to process entire sequences in parallel rather than sequentially.

The core components of the transformer architecture include:
\begin{itemize}
    \item \textbf{Self-attention mechanisms}: Allow the model to assess the importance of different words within a given context. By computing weighted sums of all positions in a sequence, with weights determined by query-key interactions, the model can focus on relevant information while ignoring irrelevant parts.
    $$Attention (Q, K, V) = softmax\left(\frac{QK^T}{\sqrt{d_k}}\right)V$$
    where \( Q \) is the query matrix, \( K \) is the key matrix, \( V \) is the value matrix, and \( d_k \) is the dimension of the keys.

    \item \textbf{Multi-head attention}: Enables the model to attend to information from multiple representation subspaces simultaneously.
    $$MultiHead(Q, K, V) = Concat(head_1, \dots, head_h)W^O$$
    where \( head_i = Attention(QW_i^Q, KW_i^K, VW_i^V) \) and \( W_i^Q, W_i^K, W_i^V \) are learned projection matrices.

    \item \textbf{Feed-forward neural networks}: Apply non-linear transformations to the attention outputs.
    \item \textbf{Layer normalization}: Helps stabilize and accelerate the training process.
    $$LayerNorm(x) = \omega \odot  \frac{x - \mu}{\sqrt{\sigma^2 + \epsilon}} + \beta$$

    where \( \mu \) and \( \sigma^2 \) are the mean and variance of the input, \( \epsilon \) is a small constant for numerical stability, and \( \omega \) and \( \beta \) are learnable parameters.
    

    \item \textbf{Positional encoding}: Introduces information about the order of tokens in a sequence. It is tipically encoded using sine and cosine functions of different frequencies.
    $$PE_{(pos, 2i)} = sin\left(\frac{pos}{10000^{\frac{2i}{d_{model}}}}\right)$$  
    $$PE_{(pos, 2i+1)} = cos\left(\frac{pos}{10000^{\frac{2i}{d_{model}}}}\right)$$
    where \( pos \) is the position and \( i \) is the dimension index.

\end{itemize}

Notable examples of LLMs include OpenAI's GPT-3 \cite{gpt3Paper}, Google's BERT \cite{bertPaper} and T5 \cite{t5GooglePaper}, and Meta's RoBERTa \cite{robertaPaper}. These models have set new benchmarks across a variety of NLP tasks such as text classification, machine translation, and summarization. While all are built on the transformer architecture, they differ in their design choices and training objectives. For instance, BERT \cite{bertPaper}, an encoder-only transformer, uses a masked language modeling objective, where random tokens in a sentence are masked and the model learns to predict them using surrounding context. In contrast, GPT-3 \cite{gpt3Paper}, a decoder-only transformer, follows an autoregressive training strategy, generating one token at a time based on previously generated tokens.

The theoretical foundation of LLMs lies in probabilistic language modeling. Given a sequence of tokens \( x = (x_1, x_2, \dots, x_n) \), a language model aims to estimate either the joint probability \( P(x) \) or the conditional probability of the next token \( P(x_n \mid x_1, \dots, x_{n-1}) \). The model is trained to assign higher probabilities to sequences that are more likely to appear in natural language, based on patterns learned from large-scale datasets. This is made possible by the transformer's use of self-attention and feed-forward layers, which together capture complex dependencies across tokens. In particular, positional encodings and multi-head attention enable LLMs to model long-range relationships—something that earlier architectures like RNNs and LSTMs struggled to achieve effectively.


% TODO: complete this subsection
% \subsection{LLM decoding methods}
% \cite{Shi2024DecodingMethods}, \cite{Minaee2025LLMSurvey}, \cite{Qin2025DynamicWidthSpeculativeBeamDecoding}

\subsection{Supervised Fine-Tuning}

% * SFT
Supervised fine-tuning (SFT) of large language models (LLMs) is a technique used to adapt pre-trained models on large corpora to specific tasks, improving their performance and alignment with human preferences or task requirements \cite{Wang2023AligningLargeLanguageModels}. This process involves training the model on a labeled dataset, where each input is paired with a corresponding output.

% * Instruction Tuning
A common variant of SFT is instruction tuning, which trains the model on supervised datasets containing (instruction, output) pairs \cite{Wei2022FinetunedLMZeroShot}. This method improves the model’s zero-shot performance by shifting its behavior from next-token prediction to better understanding and following human directions \cite{Zhang2024InstructionTuningLLM}. 

% ? TODO: Poner esto o no?: In the context of crime data, instruction tuning can help the model better interpret domain-specific terminology and generate more accurate, context-aware responses.
% ! TODO: formula de next token prediction en LLMs

% Supervised fine-tuning is a process where a pre-trained language model is further trained on a labeled dataset specific to a task. This step refines the model's weights to improve performance on the desired application. 
% \cite{Mishra2022CrossTaskGeneralization}, \cite{Ouyang2022TrainingLMsIT}


\subsection{LoRA (Low-Rank Adaptation)}     % TODO: rename to Parameter-Efficient Fine-Tuning


LoRA \cite{Hu2021LoRA} is a fine-tuning method designed to efficiently adapt large language models (LLMs) by minimizing computational overhead. Instead of updating all model parameters, which can number in the hundreds of billions for modern LLMs, LoRA introduces trainable low-rank matrices into existing weight layers, such as attention or feed-forward layers. These matrices encode task-specific adjustments while keeping the pre-trained model weights unchanged, significantly reducing the number of parameters that need to be trained.

Mathematically, LoRA approximates the weight update $\Delta W$ using a low-rank decomposition $\Delta W = A B$, where $A \in \mathbb{R}^{d \times r}$ and $B \in \mathbb{R}^{r \times k}$, with $r \ll \min(d,k)$. This approach greatly lowers memory and computational requirements, making it particularly suitable for fine-tuning on devices with limited resources.

LoRA is highly modular and can be seamlessly integrated into transformer-based architectures without requiring structural modifications. It can also be combined with other optimization techniques, such as quantization or pruning, to further enhance efficiency. Studies have demonstrated that LoRA achieves performance comparable to or better than full fine-tuning across various tasks, while modifying less than 1.% of the model's parameters.

Extensions of LoRA, such as QLoRA (Quantized LoRA) and AdaLoRA (Adaptive LoRA) \cite{Dettmers2023QLoRA, Zhang2023AdaLora}, have further refined its capabilities. For instance, QLoRA enables fine-tuning on models quantized to 4 bits \cite{Dettmers2023QLoRA}, facilitating efficient deployment on consumer-grade hardware.

Thanks to its efficiency and flexibility, LoRA has become a popular choice in open-source LLM projects, especially for scenarios with limited computational resources or strict privacy requirements.


\begin{figure}[hbtp]
  \centering
  \includegraphics[width=0.29\textwidth]{images/lora.pdf}
  \caption{LoRA}
  \label{fig:reparam}
\end{figure}

\subsection{Tool Integrated Reasoning}

Tool-integrated Reasoning Agents (ToRA) \cite{Gou2024ToRA} offer a conceptual framework for augmenting large language models (LLMs) by enabling them to utilize external tools during the reasoning process. Rather than relying solely on internal language generation, ToRA integrates natural language inference with symbolic computation or code execution, treating reasoning as a hybrid process. This approach is inspired by dual-process cognitive theories—System 1 (intuitive) and System 2 (analytical)—and positions the LLM as a decision-making controller that determines when and how to delegate tasks to tools such as symbolic solvers, code interpreters, or retrieval systems.

This paradigm is exemplified by models like AIMO2 \cite{Moshkov2025AIMO2}, which propose modular agent architectures for tool interaction, and MuMathCode \cite{Yin2024MuMathCode}, which highlights the benefits of program synthesis and iterative self-correction for improving mathematical reasoning. ToRA adopts a similar methodology, leveraging training on tool-use trajectories and imitation learning to teach models how to incorporate intermediate tool outputs into reasoning workflows. Additionally, NuminaMath \cite{Li2024NuminaMath, Fleureau2024NuminaMath} introduces symbolic memory modules that align with ToRA’s goal of maintaining coherent multi-step reasoning through persistent access to verified tool outputs.

% Tool Integrated Reasoning enhances the reasoning capabilities of language models by integrating external tools, such as APIs or specialized algorithms, into the generation process. In the context of crime data chatbots, this approach allows the model to query real-time crime databases or perform geospatial analysis to provide users with accurate and actionable information about potential dangers.




\section{Prompting}

Basically prompting is a technique used to guide the behavior of LLMs by providing them with specific instructions or context. 
This can be done through various methods, such as zero-shot prompting, few-shot prompting, and chain-of-thought prompting. 
Each method has its own advantages and disadvantages, depending on the task at hand and the desired outcome.


\subsection{Prompt Engineering}

% https://www.promptingguide.ai/ cite from here 
Prompt engineering is an emerging field focused on crafting and refining prompts to make the most effective use of language models (LMs) 
across diverse applications and research areas. It plays a crucial role in enhancing our understanding of the strengths and limitations of LLMs. 
Beyond simply writing prompts, prompt engineering involves a broad set of techniques essential for building with, interacting with, and expanding the capabilities of LLMs. 
It also contributes to improving model safety and enables the integration of specialized knowledge and functionalities into LLM-based systems

The following methods are some of the most common techniques and used in our work:

\begin{itemize}
    \item \textbf{Zero-shot prompting}: Involves providing the model with a task description or question without any examples. The model is expected to generate a response based solely on its pre-existing knowledge and understanding of the task.
    \item \textbf{Few-shot prompting}: As mentioned by \cite{gpt3Paper}, this technique involves providing the model with a few examples of the desired output format or task. This helps the model understand the context and generate more accurate responses. A related approach is "one-shot prompting," in which the model is given only a single example.
    \item \textbf{Chain-of-thought prompting}: Firstly introduced by \cite{chainofthought2023}, encourages the model to generate intermediate reasoning steps before arriving at a final answer. This approach has been shown to improve performance on complex tasks, such as arithmetic reasoning and logical inference.
    % TODO: PAL https://www.promptingguide.ai/techniques/pal
\end{itemize}




% \section{Retrieval Augmented Generation (RAG)}

% Retrieval-Augmented Generation (RAG), first introduced by \cite{RAG2021}, is a framework that combines the strengths of retrieval-based and generative approaches for question answering. Its core idea is to enhance the generative capabilities of language models by incorporating relevant information retrieved from external knowledge sources, rather than relying solely on parametric knowledge (i.e., information encoded in the model's weights).

% The RAG architecture consists of two primary components: 
% \begin{itemize}
     
%     \item \textbf{Retriever}: This component retrieves relevant documents or sentences from a large corpus based on the input query. It typically employs models such as BM25 \cite{bm25Paper} or dense retrieval methods to identify the most relevant information. 
%     \item \textbf{Generator}: Once the retriever has collected the relevant documents, the generator—usually a transformer-based language model—takes them as input to produce a coherent and contextually appropriate response. The generator can be fine-tuned for specific tasks to further improve its performance. 
% \end{itemize}

% In recent years, RAG has gained considerable attention in the NLP community for its ability to generate high-quality responses. This has spurred the development of new techniques to improve both retrieval precision and generation quality. For instance, \cite{modularRAG2024} introduces a modular RAG framework that enables the integration of diverse components across different stages of the pipeline. These include techniques such as query expansion, reformulation, and transformation in the pre-retrieval phase, as well as the use of LLMs as judges in the post-retrieval phase to evaluate if the response is enoughly complete.
% This modular design encourages flexibility and supports experimentation with novel combinations and architectural variations.



\section{Metrics}

\subsection{Pass@K}

Pass@k is a metric that evaluates the probability of a model generating a correct answer within the top-k responses. It is particularly useful for assessing the performance of language models in tasks where multiple answers can be generated, such as question answering or text generation \cite{Levi2024SimpleModelInferenceScalingLaws}. The metric is defined as:

\begin{equation}
    Pass@k = \frac{1}{N} \sum_{i=1}^{N} \mathbb{I}(y_i \in \hat{y}_i^{(1:k)})
\end{equation}

where \(N\) is the number of samples, \(y_i\) is the ground truth answer for sample \(i\), and \(\hat{y}_i^{(1:k)}\) is the set of the top-k generated answers for sample \(i\). The indicator function \(\mathbb{I}\) returns 1 if the ground truth answer is in the top-k responses, and 0 otherwise.

\subsection{Majority@K}

Majority@k is a metric that evaluates if the group with the majority of the top-k generated answers is similar to the ground truth answer. It is particularly useful for assessing the performance of language models in tasks where multiple answers can be generated, such as question answering or text generation \cite{Wang2023SelfConsistency}. The metric is defined as:
\begin{equation}
    Majority@k = \frac{1}{N} \sum_{i=1}^{N} \mathbb{I}(y_i \in \text{majority}(\hat{y}_i^{(1:k)}))
\end{equation}
where \(N\) is the number of samples, \(y_i\) is the ground truth answer for sample \(i\), and \(\hat{y}_i^{(1:k)}\) is the set of the top-k generated answers for sample \(i\). The indicator function \(\mathbb{I}\) returns 1 if the ground truth answer is in the majority of the top-k responses, and 0 otherwise.


\subsection{Code-BLEU}

CodeBLEU is a specialized metric designed to evaluate the quality of code generation by comparing machine-generated code against one or more reference implementations. It extends the traditional BLEU metric \cite{Papineni2002BLEU}, commonly used in natural language processing for tasks such as machine translation—by incorporating programming-specific characteristics such as syntax, structure, and semantic correctness \cite{Ren2020CodeBLEU}. This makes CodeBLEU better aligned with the nuanced requirements of programming languages.

The metric is formulated as a weighted linear combination of four complementary components:
\begin{equation}
    \text{CodeBLEU} = \alpha \cdot \text{BLEU} + \beta \cdot \text{BLEU}_{\text{weight}} + \gamma \cdot \text{Match}_{\text{AST}} + \delta \cdot \text{Match}_{\text{DF}}
\end{equation}
Here, \(\text{BLEU}\) is the standard BLEU score, \(\text{BLEU}_{\text{weight}}\) is a weighted variant that accounts for the relative importance of different n-grams, \(\text{Match}_{\text{AST}}\) measures syntactic similarity based on abstract syntax trees (ASTs), and \(\text{Match}_{\text{DF}}\) evaluates semantic similarity through data flow analysis. The coefficients \(\alpha\), \(\beta\), \(\gamma\), and \(\delta\) are tunable hyperparameters that balance the influence of each component.

The BLEU component is computed as:

\begin{equation}
    \text{BLEU} = BP \cdot \exp\left(\sum_{n=1}^{N} w_n \cdot \log p_n\right)
\end{equation}

where \(BP\) is the brevity penalty, \(w_n\) denotes the weight assigned to n-grams of length \(n\), and \(p_n\) is the precision of those n-grams in the generated code relative to the reference code.



% \subsection{Exact Match (EM)}

% \subsection{Accuracy}

% \subsection{Precision}

% \subsection{Recall}

% \subsection{F1}

% \subsection{BERTScore}

% \subsection{Hit Rate}

% \subsection{Mean Reciprocal Rank (MRR)}

\chapter{ STATE OF THE ART}

% TODO: introduction

\section{Large-Language Models for Urban Scenarios}

Recent advances show that LLMs are increasingly being adapted to address urban computing tasks. Three recent approaches illustrate how these models can be used for both forecasting urban phenomena and orchestrating multiple models to tackle complex urban tasks.

\cite{Li2024UrbanGPT} proposes UrbanGPT, a spatio-temporal LLM for forecasting urban dynamics such as traffic flows and crime rates. The model receives spatial and time series information through the prompt, then employs a spatio-temporal dependency encoder and a lightweight alignment module to project these representations into the LLM's latent space, achieving performance on par with or surpassing state-of-the-art models in multiple datasets. 

Complementing this, \cite{Jiang2024UrbanLLM} leverage the agentic capabilities of LLMs to decompose urban-related queries into structured sub-tasks (e.g., forecasting, anomaly detection, POI recommendation, etc). This approach, termed UrbanLLM, assigns each sub-task to a specialised model from a curated model zoo, and integrates the results into a unified response.


Recent research from Google has explored the use of foundation models for geospatial reasoning \cite{2025GoogleGeospatialReasoning}. It introduces an agentic workflow powered by Gemini to assist users in tasks such as visualizing pre and post disaster scenarios or conducting damage assessments. Their approach integrates diverse modalities: maps, weather data, and satellite imagery, and highlights the need for foundation models capable of aligning heterogeneous spatial information . Together with UrbanGPT's forecasting focus and UrbanLLM's model orchestration, this work reflects a growing trend toward multimodal LLM-driven systems for urban scale analysis and decision making.

Another study from Google introduces Visual Chronicles \cite{Deng2025VisualChronicles}, a multimodal LLM-based system designed to identify and describe frequently occurring visual changes across urban environments using a dataset provided by Google Street View imagery. It leverages a vast collection of geolocated, timestamped images to identify trends without requiring labeled training data. To overcome the limitations of MLLMs in processing such massive datasets, the authors design a scalable pipeline that enables efficient retrieval, comparison, and semantic analysis of visual patterns across both space and time.
% \cite{Deng2025VisualChronicles} highlight the system's ability to uncover temporal patterns using MLLMs, capitalizing on their open-ended semantic understanding capabilities. Given that the datasets are several orders of magnitude too large for an MLLM to process as context, the system employs innovative techniques to manage and analyze the data effectively.

% There are more LLMs on Urban Applications: See UrbanLLM paper
% \cite{Zhang2023GeoGPT}



\section{RAG Techniques for Complex Data}

% * Introduction
Retrieval-Augmented Generation (RAG) frameworks aim to enhance LLMs by integrating external sources of knowledge, such as structured databases, time series, or knowledge graphs. Recent research has extended RAG beyond textual documents to support spatial, temporal, and graph-based data retrieval.

% * Spatial RAG
\cite{Yu2025SpatialRAG} extends RAG to spatial tasks by integrating sparse spatial retrieval (SQL-based queries) with dense semantic retrieval (LLM-based similarity). Their method introduces three preprocessing steps to help the LLM generate complete and executable SQL queries, addressing its limitations in query formulation.
% , addressing the common challenge of LLMs struggling to construct such queries directly from user input.

% * RAG on temporal data
In the temporal domain, \cite{Yang2024TimeRAG} apply RAG to the context of time series forecasting using Dynamic-Time Warping (DTW) as a distance metric to retrieve similar waveforms and trends, given a time serie as a query. The retrieved information is then utilized to improve the LLM forecasting accuracy. 

% * RAG on graphs
Other works, combine RAG techniques and hybrid approaches to address question-answering over textual knowledge graphs. One such example is \cite{He2024GRetriever}, who introduce G-Retriever, a flexible QA framework for knowledge graphs that incorporates a RAG into its pipeline. The framework separates node entities and edge information into two distinct embedding spaces. Using cosine similarity, it retrieves the most relevant nodes and edges for the query and reconstructs the subgraph using the Prize-Collecting Steiner Tree (PCST) algorithm. The final answer is generated by a hybrid GNN-LLM, which processes the retrieved subgraph both as text in the query prompt and through a graph encoder aligned with the LLM's token space.

Building on this idea of graph-based retrieval, \cite{Hu2024GRAG} map textual subgraphs directly to an embedding space, allowing for the retrieval of relevant subgraphs based on their semantic similarity to the query. Then applied techniques to merging and pruning the retrieved subgraphs to improve the quality of the final answer. 

Recent work from \cite{Guo2024LightRAG} propose LightRAG, a fully prompt-driven framework that extracts knowledge graphs, generates keywords at multiple granularities, retrieves from both vector and graph indexes, and supports fast incremental updates.

% TODO: \cite{Edge2025GraphRAG}
% \cite{Xiao2024TimeRAG}
% \cite{Chen2025KGRAGSurvey}



\section{Open Crime Datasets}

Among the most popular datasets for crime analysis are the Chicago Crime dataset \cite{ChicagoDataset} and the New York City Crime dataset \cite{NYCDataset}. These datasets contain detailed records of reported crimes, including information on the type of crime, location, time, and other relevant attributes. They have been widely used in various research studies and applications related to crime prediction, analysis, and visualization.

Beyond the United States, similar efforts have been made in Latin America. In Brazil, for example, crime datasets are made publicly available at the state level through open data platforms. These resources have supported a range of research projects, including \cite{Garcia2022CriPAV} and \cite{Waqar2025CrimePredictionGNN}, which process and analyze regional crime patterns or develop predictive models.

More recently, large-scale initiatives have emerged in Asia. \cite{Zhang2025CrimeDatasetChina} introduces a large-scale crime dataset from China, comprising approximately 1 million records. The dataset spans 31 provincial-level administrative regions, 222 city-level divisions, and 548 county (district)-level jurisdictions across mainland China. Unlike the structured records in the aforementioned datasets, this resource was constructed by extracting crime information from unstructured judicial documents using LLMs, enabling broader geographic and semantic coverage. Additionally, it includes detailed fields such as case descriptions, victim and defendant information, and final judgments, offering more possibilities for anlysis and research.



\section{Crime-Data Visualization Tools}

In the context of crime data analysis, several visualization systems have been proposed to support pattern recognition, hotspot identification, and urban context interpretation. These tools typically integrate geospatial data with interactive visual analytics techniques to assist expert users in understanding complex crime patterns.

Early tools like CrimeVis \cite{Silva2017CrimeVisAI}, focused on interactive exploration accross police districts (DPs) brushing-and-linking techniques. Later systems extended this groundwork by incorporating advanced feature engineering and machine learning techniques. CrimAnalyzer \cite{Garcia2021CrimAnalyzer} proposed a Non-negative Matrix Factorization (NMF) based technique to identify hostpots. Furthermore, CriPAV \cite{Garcia2022CriPAV} incorporates autoencoders to embed and cluster hotspots, facilitating the analysis of the relationship between crime and urban features.


% \cite{Garcia2021CrimAnalyzer} propose CrimAnalyzer, that relies on Non-negative Matrix Factorization (NMF) to identify hostpots.
% \cite{Garcia2022CriPAV} propose CriPAV a visualization system to assist experts to figure out the relation between crime and urban features, using autoencoders to generate embeddings of hotspots to cluster them. 


% TODO Include: \cite{Garcia2020MiranteAV} introduces Mirante, a crime mapping visualization system that allows pattern analysis in a street-level scale.
% \cite{Salah2022BigCDVis} 



\section{NLP in Data Visualization}

Eviza \cite{Setlur2016Eviza} convert natural language input to filters applied to visualizations. 

\cite{Setlur2016Eviza} introduces Eviza, a natural language interface for visual data analysis, leveraging a probabilistic grammar-based approach with predefined syntactic rules. The system incorporates a template-based autocompletion feature to provide users with contextual suggestions, enabling an interactive conversation with existing visualizations. Tested on geographic datasets, such as earthquake data in the US, the interface enhances user interaction by implementing language pragmatics through a finite state machine. However, Eviza faced challenges in recognizing all parts of long and complex queries and struggled with certain grammatical constructs, highlighting the need for more robust natural language processing techniques in visualization systems.


\cite{Wu2024LLMVis}


Early research on natural language interfaces for data visualization has explored how users can interact with visual content through conversational input.

\cite{Liu2024NLDriven} introduce a framework for controlling data visualizations through natural language. Their approach centers on two key components: a natural language-to-task translator and a visualization manipulation parser. The translator, based on a fine-tuned T5 model, maps user queries into a hierarchical structure of tasks, which are then interpreted to apply manipulation operations over existing visualizations.


\cite{Luo2022NL2Vis}

\cite{Narechania2021NL4DV}

\cite{Liu2021ADVISor}

% \section{KGQA Datasets}
% ExplaGraphs, WebQSP, SceneGraphs


% \section{Question Answering on Knowledge Graphs}
% \cite{Dai2024QASTKG}


% \section{GNN-LLM}
% \cite{He2024GRetriever}, \cite{Perozzi2024GraphToken}, \cite{Fatemi2023GraphEncoding}


% ? DL-based crime prediction encounters significant obstacles, such as concerns about fairness, accountability, transparency, interpretability, security, and privacy \cite{Ersoz2025CrimePredictionXAISurvey}.
% ? Inexeperienced users have difficulties using database query languages like SQL to express their data needs \cite{Setlur2016Eviza}. 
% ? Natural language can be ambiguous due to syntactic and semantic variations between the user's mental model and the system's mental model \cite{Setlur2016Eviza}.

\chapter{ METHODOLOGY}

% In this chapter we present the methodology used in this research. We begin by outlining the research objectives, followed by a detailed description of the proposed architecture. We then discuss the datasets utilized in our experiments, including their characteristics and relevance to our study. Finally, we describe the evaluation metrics and benchmarks employed to assess the performance of our proposed approach.

% - Actualmente, en el estado del arte no ha sido explorada arquitecturas de RAG orientada directamente a la búsqueda de información espacio-temporal y utilizar esta información para generar respuestas a preguntas relacionadas con la criminalidad de una localidad en un determinado periodo de tiempo. 
Recent advancements in Retrieval-Augmented Generation (RAG) have shown great promise across various domains \cite{Yu2025SpatialRAG, Yang2024TimeRAG, He2024GRetriever, Hu2024GRAG, Guo2024LightRAG}; however, none of these architectures are specifically tailored for spatio-temporal information retrieval and reasoning over crime data. Existing approaches typically focus on textual or knowledge graph-based sources, leaving a key research gap for systems capable of handling dynamic urban crime contexts across space and time.

% - No existe una herramienta de visualización geográfica que permita a los usuarios hacer consultas mediante lenguaje natural 
In parallel, numerous visualization tools have been developed to support crime data analysis \cite{Garcia2022CriPAV, Salah2022BigCDVis, Silva2017CrimeVisAI, Garcia2020MiranteAV, Garcia2021CrimAnalyzer}, and recent works attempt to bridge natural language interfaces with visual analytics \cite{Liu2024NLDriven}. Yet, none of these efforts fully integrate geographic crime data querying through natural language while also offering intuitive, interactive visualizations. This highlights a missed opportunity to democratize access to urban crime insights.

% - Se ha explorado el uso de LLMs para el Q\&A de datos espaciales; sin embargo, ....
Moreover, while large language models (LLMs) have been used for spatio-temporal question answering, current pipelines still face limitations. For instance, \cite{Wei2024TourLLM} embed domain knowledge directly into the model via fine-tuning, which reduces transparency and flexibility. Google's recent work \cite{2025GoogleGeospatialReasoning} proposes LLM-based reasoning over geospatial data, but without a focus on urban safety or crime-specific tasks. Similarly, \cite{Jiang2024UrbanLLM} and \cite{Li2024UrbanGPT} apply LLMs to urban computing, but rely on pre-embedded data in prompts, bypassing interactive user-driven retrieval.

This section presents our proposed architecture, designed to overcome these limitations by enabling spatio-temporal crime data analysis and visualization through natural language interaction.

% To address these limitations, we propose a modular architecture that leverages a multi-stage LLM pipeline with hybrid retrieval strategies to support natural language interactions over spatio-temporal crime data. Our method enables users to ask localized, time-sensitive questions and receive coherent, explainable responses, complemented by dynamic visualizations. The following sections describe each component of the proposed system.

% TODO: mandar estas ideas a justificacion

\section{Methodological Proposal}

The proposal is divided into two phases. The first phase focuses on the development of a prototype that integrates a hybrid retrieval mechanism with an LLM-based chat interface. This prototype will be evaluated through user studies to assess its usability and effectiveness in answering spatio-temporal crime questions. The second phase aims to enhance the system by generating synthetic datasets and fine-tuning the selected LLMs based on the generated data. This approach will allow us to improve the model's performance and adapt it to specific crime-related tasks.



\section{Datasets}

% DONE: Poner que tipos de preguntas estadisticas se van a abordar, tabla de clasificacion de pregunta con su ejemplo (clasificacion, ejm de plantilla de pregunta) para el dataset de China, hazlo una tabla

We base our prototype on the dataset introduced by \cite{Zhang2025CrimeDatasetChina}, which provides a large-scale, open-access repository of nearly one million criminal court records across China. From this dataset, we construct a benchmark of spatio-temporal statistical questions designed to evaluate the performance of our LLM-based system in crime data exploration. Inspired by prior work on question classification for temporal knowledge graphs \cite{Saxena2021TemporalKGQA}, as well as tourism and spatial reasoning benchmarks \cite{Contractor2020QATourism, Dai2024QASTKG}, we define a taxonomy of question types that reflects the analytical goals of urban crime investigation.

Table~\ref{tab:dataset_questions} summarizes the types of questions we support, along with representative templates and instantiated examples using the Chinese crime dataset.


\begin{table}[H]
    \centering
    \caption{Question type examples supported over the spatio-temporal crime dataset}
    \label{tab:dataset_questions}
    \begin{tabular}{|p{4.5cm}|p{10cm}|}
    \hline
    \textbf{Category} & \textbf{Question Template Example} \\
    \hline
    \textbf{Simple time reasoning} & How many crimes occurred on \textless Time Entity\textgreater? \\
    \hline
    \textbf{Spatial aggregation} & How many incidents occurred in \textless Spatial Entity\textgreater? \\
    \hline
    \textbf{Spatio-temporal filtering} & How many crimes happened in \textless Spatial Entity\textgreater during \textless Time Entity\textgreater? \\
    \hline
    \textbf{Before/After comparison} & Did crime increase in \textless Spatial Entity\textgreater after \textless Time Point\textgreater? \\
    \hline
    \textbf{First/Last occurrence} & When was the last crime reported in \textless Spatial Entity\textgreater? \\
    \hline
    \textbf{Most affected area} & What is the most crime-prone \textless Spatial Level\textgreater during \textless Time Period\textgreater? \\
    \hline
    \textbf{Location-based correlation} & How does crime frequency vary between \textless Entity 1\textgreater and \textless Entity 2\textgreater? \\
    \hline
    \textbf{Intersection or routing} & What streets intersect with \textless Street Name\textgreater? \\
    \hline
    \end{tabular}
\end{table}
    

% This taxonomy will guide the construction of training and evaluation prompts in both phases of the project. In Phase 1, templates are used to manually generate synthetic question-answer pairs. In Phase 2, these templates will be programmatically expanded using the dataset's metadata and LLM-based generation strategies.


\subsection{Phase 1: Expected PFC3}

The following components structure this phase:

\begin{itemize}
    \item \textbf{Dataset Question Generation}: Utilizing question templates such as those proposed in \cite{Contractor2020QATourism} and \cite{Dai2024QASTKG}, we will generate spatio-temporal statistical questions derived from the dataset presented in \cite{Zhang2025CrimeDatasetChina}.
    \item \textbf{Hybrid Retrieval Mechanism}: Drawing on methodologies outlined in works like \cite{Guo2024LightRAG}, a hybrid RAG mechanism will be implemented, integrating vector-based and query-based retrieval approaches to enhance accuracy.
    \item \textbf{Model Selection and Prompting}: Open-source LLMs, like Llama3 \cite{Grattafiori2024Llama3} and Qwen2.5 \cite{Qwen2025Qwen2.5} series, will be evaluated for their effectiveness in generating responses to the formulated questions.
    \item \textbf{Chat Implementation and Feedback Loop}: A chat-based interface will be developed to preprocess user queries and generate responses. This interface will incorporate geospatial visualization tools to provide users with visual feedback. Additionally, user studies will be conducted to evaluate the usability and effectiveness of the interface.
\end{itemize}

% We will also conduct user studies to assess the usability and effectiveness of the chat interface.


\subsection{Phase 1: Pipeline}

The proposed pipeline, illustrated in Figure~\ref{fig:proposal_f1}, describes the architecture of the first prototype. The system is designed to process natural language queries about crime occurrences across space and time, combining language models with a hybrid retrieval strategy and visual feedback mechanisms.

%  such as: “Has robbery increased in the northern districts compared to last year?” 
The architecture begins with an \textbf{LLM-powered chat interface}, where users can ask statistical-spatio temporal queries. This initial query is processed through a \textbf{query decomposition module}, which splits the question into street-level subqueries that the selected model can handle.

% , city sectors
These subqueries are routed through a \textbf{hybrid retrieval mechanism}, inspired by \cite{Guo2024LightRAG}, which is divided in two \textbf{LLM extractors}. The first, the \textbf{Entity Extractor} extract named entities (e.g., street names) using a dense retriever based on multilingual sentence embeddings and a sparse retriever using BM25. While the \textbf{Time Extractor} extracts temporal references (e.g., “Q1 2024”, “last month”). With this information we retrieve the spatio-temporal data from the dataset, which is a collection of crime records. These records are then passed to a \textbf{data parsing module}, which processes the data and pass to the prompt as a contextual information. 

Once the context has been aligned, each subquery is processed by an LLM, which generates a set of answers. These answers are then passed to a \textbf{summarization module} that produces the final response. This response is further enhanced with \textbf{visual elements}, such as a map that highlights the queried streets.

The \textbf{query decomposition module} and \textbf{summarization module} was inspired by the bottom-up approach presented by \cite{Deng2025VisualChronicles}.

\begin{figure}[H]
    \centering
    \includegraphics[width=\textwidth]{images/PFC3.drawio.png}
    \caption{Proposed Pipeline - Phase 1}
    \label{fig:proposal_f1}
\end{figure}

\subsection{Phase 2: Expected Next Semester}

\begin{itemize}
    \item \textbf{Synthetic Dataset Generation}: The capabilities of proprietary LLMs will be leveraged to generate synthetic answers (\cite{Nvidia2024KaggleMath}, \cite{Liu2024NLDriven}) based on the formulated questions and the retrieval mechanism. The generated chain-of-thought (CoT) reasoning and associated code will be stored for fine tune the selected LLMs.
    \item \textbf{Model Fine-tuning and Evaluation}: The selected LLMs will be fine-tuned using the expanded dataset, and their performance will be assessed using metrics such as BLEU, ROUGE, BertScore, and METEOR.
\end{itemize}


% \section{Benchmarks}

% \subsection{Estructura de los \textit{datasets}}

% \section{Entrenamiento/Optimización del modelo}

% \section{Evaluación del modelo}


% \lstdefinestyle{pythoncode}{
  language=Python,
  basicstyle=\tiny\ttfamily,
  backgroundcolor=\color{gray!5},
  keywordstyle=\color{blue},
  commentstyle=\color{green!50!black},
  stringstyle=\color{orange},
  numberstyle=\tiny\color{gray},
  numbers=left,
  stepnumber=1,
  numbersep=8pt,
  frame=single,
  breaklines=true,
  captionpos=b,
  showstringspaces=false,
  tabsize=4,
  baselinestretch=0.8,
  lineskip=-1pt
}
% Ajuste opcional del formato de captions
\captionsetup[lstlisting]{font=small,labelfont=bf}



\chapter{ EXPERIMENTS AND RESULTS}

This chapter presents the outcomes derived from applying the methodology detailed in the preceding chapter. We begin by specifying the hyperparameters employed during the training phase. Next, we report the results from the evaluated models, covering both quantitative and qualitative analyses. Lastly, we discuss the significance of these findings and include case studies to demonstrate the practical relevance of our results.

\section{Hyperparameters}


The selection of key hyperparameters for training was guided by empirical observation and best practices in fine-tuning large language models:

\begin{itemize}
  \item \textbf{Warmup Ratio:} We initially set the warmup ratio to 0.03; however, this led to instability during the early stages of training, with noticeable spikes in the loss. Increasing the warmup ratio to 0.05 significantly improved training stability, consistent with findings in transformer-based models such as RoBERTa, where extended warmup periods are known to facilitate smoother convergence \citep{liu2019robertarobustlyoptimizedbert}.
  \item \textbf{Weight Decay:} To mitigate overfitting and enhance generalization, especially given the relatively small size of our dataset, we applied a weight decay of 0.05. This choice aligns with established practices in training deep neural networks on limited data, where appropriate regularization is crucial for model robustness \citep{brainacgan}.
  \item \textbf{Learning Rate:} The learning rate was initially set to 2e-3, but this configuration resulted in poor convergence during training. Considering that we employed LoRA for fine-tuning the Llama model, we reduced the learning rate to $1.5e^{-4}$. This lower rate aligns with commonly adopted values for adapting pre-trained large language models using parameter-efficient techniques, and it yielded significantly improved convergence behavior \citep{zhou2024automixqselfadjustingquantizationhigh}.
\end{itemize}

The other hyperparameters (such as LoRa range and batch size) were chosen according to hardware limitations. Full detail in Table \ref{tab:hyperparameters}.

\begin{table}[hbtp]
\centering
\small
\begin{tabular}{ll}
  \toprule
  \textbf{Hyperparameter} & \textbf{Value} \\
  \midrule
  LoRA rank ($r$) & 32 \\
  LoRA alpha & 32 \\
  LoRA dropout & 0.05 \\
Batch size (per device) & 32 \\
Gradient accumulation steps & 8 \\
Max sequence length & 3,000 tokens \\
Training epochs & 7 \\
Learning rate & $1.5e^{-4}$ \\
Optimizer & \texttt{adamw\_torch\_fused} \\
Scheduler & \texttt{cosine\_with\_restarts} \\
Warmup ratio & 0.05 \\
Weight decay & 0.05 \\
Quantization & 4-bit (NF4, double quant) \\
\bottomrule
\end{tabular}
\caption{Fine-tuning Hyperparameters for ChinaCrimesQACode}
\label{tab:hyperparameters}
\end{table}

\section{Quantitative Results}
\begin{table}[hbtp]
\centering
\footnotesize
\setlength{\tabcolsep}{2pt}
\begin{tabular}{l|ccc|ccc|ccc|ccc}
\toprule
\textbf{Metric} & \multicolumn{3}{c|}{\textbf{Fine-tuned}} & \multicolumn{3}{c|}{\textbf{Base}} & \multicolumn{3}{c|}{\textbf{o4-mini}} & \multicolumn{3}{c}{\textbf{DeepSeekV3}} \\
 & \multicolumn{3}{c|}{\textbf{Llama3}} & \multicolumn{3}{c|}{\textbf{Llama3}} & \multicolumn{3}{c|}{} & \multicolumn{3}{c}{} \\
\cmidrule(lr){2-4} \cmidrule(lr){5-7} \cmidrule(lr){8-10} \cmidrule(lr){11-13}
 & \textbf{Mean} & \textbf{Std} & \textbf{Med} & \textbf{Mean} & \textbf{Std} & \textbf{Med} & \textbf{Mean} & \textbf{Std} & \textbf{Med} & \textbf{Mean} & \textbf{Std} & \textbf{Med} \\
 \midrule
Percent Error@k\,($\downarrow$)  & 0.089 & 0.195 & \textbf{0.000} & 0.220 & 0.315 & \textbf{0.000} & \textbf{0.003} & \textbf{0.014} & \textbf{0.000} & 0.025 & 0.060 & \textbf{0.000} \\
Pass@k\,($\uparrow$)           & \textbf{0.784} & \textbf{0.412} & \textbf{1.000} & 0.267 & 0.442 & 0.000 & 0.303 & 0.459 & 0.000 & 0.202 & 0.401 & 0.000 \\
Pass\textasciicircum k\,($\uparrow$)     & \textbf{0.581} & \textbf{0.417} & \textbf{0.667} & 0.109 & 0.235 & 0.000 & 0.063 & 0.175 & 0.000 & 0.051 & 0.161 & 0.000 \\
CodeBLEU@k\,($\uparrow$)     & \textbf{0.403} & \textbf{0.077} & \textbf{0.400} & 0.332 & 0.049 & 0.331 & 0.370 & 0.054 & 0.375 & 0.372 & 0.061 & 0.379 \\
\bottomrule
\end{tabular}
\caption{Comparison of Evaluation Metrics: Fine-tuned Llama3, Base Llama3, o4-mini, and DeepSeekV3 (Test Set)}
\label{tab:combined_metrics}
\end{table}

Table~\ref{tab:combined_metrics} reveals that our fine-tuned Llama3-8B-Instruct model achieves superior functional performance with Pass@k of 0.784 and Pass\textasciicircum k of 0.581, substantially outperforming o4-mini (0.303 and 0.063) and DeepSeekV3 (0.202 and 0.051). This demonstrates that domain-specific fine-tuning enables the model to generate semantically correct solutions for crime data analysis tasks, prioritizing analytical accuracy over syntactic perfection.

However, the fine-tuned model exhibits higher Percent Error@k (0.089) compared to o4-mini (0.003) and DeepSeekV3 (0.025), indicating more compilation errors. This apparent contradiction suggests that general-purpose models often produce syntactically correct code that compiles successfully but fails to solve the underlying problem, while our model prioritizes domain reasoning. DeepSeekV3's elevated CodeBLEU@k (0.372) reflects stylistic similarities to our training data rather than functional superiority, as evidenced by its poor Pass@k performance.

These findings underscore the decisive value of domain-specific adaptation: the base Llama3 model without fine-tuning shows severely degraded performance (Pass@k: 0.267), while our targeted approach achieves practical problem-solving capability despite compilation imperfections. Expanding the crime-domain training dataset could address the Percent Error@k gap while maintaining superior semantic understanding, demonstrating that task-specific fine-tuning is essential for specialized analytical domains.

% Based on the observed metrics in Table \ref{tab:combined_metrics}, several conclusions emerge regarding the performance of the fine-tuned Llama3‑8B‑Instruct model versus the base o4‑mini. First, Llama3‑8B‑Instruct exhibits a higher Percent Error@k than o4‑mini and DeepSeekV3; yet it significantly outperforms o4‑mini in both Pass@k and Pass\textasciicircum k. This indicates that, despite a slightly lower rate of successful compilation, the fine‑tuned model produces more precise and semantically relevant solutions. In contrast, o4‑mini's better Percent Error@k can be misleading: it often compiles code that runs but fails to solve the problem correctly.

% These findings indicate that broadening and diversifying the crime-domain dataset is essential: a richer corpus will surface edge cases, lower the 0.233 → 0.003 Percent Error@k gap, and lift Pass@k beyond its current 0.742 vs 0.303 advantage. We selected o4-mini because its balance of accuracy, cost, and speed makes it a practical deployment benchmark, yet without domain-specific fine-tuning it still trails the task-specific precision achieved by Llama3-8B-Instruct, demonstrating the decisive value of targeted data and adaptation.

% Building on the results, we conclude that expanding and diversifying the crime‑domain dataset is imperative: a broader corpus will expose edge cases, reduce Percent Error@k, and allow the model to generalize across varied spatio‑temporal crime patterns, ultimately driving all key metrics upward. We chose o4‑mini and DeepSeekV3 as our baseline because its favorable trade‑off among accuracy, computational efficiency, and cost makes it a realistic benchmark for real‑world deployment, yet the current evaluation clearly shows that without domain‑specific fine‑tuning it cannot match the task‑specific precision of Llama3‑8B‑Instruct, underscoring the decisive impact of targeted data and adaptation.

% This discrepancy is likely rooted in architectural and training differences. o4‑mini is believed to use a Mixture‑of‑Experts (MoE) setup with approximately 40B total parameters—only about 8B of which are active per inference—whereas Llama3‑8B‑Instruct is a dense model with all 8B parameters actively contributing . While MoE architectures provide access to a broader parameter pool, this capacity advantage does not automatically translate to superior domain-specific performance. In contrast, the fully fine‑tuned Llama model has internalized domain-specific patterns—such as those needed to manipulate crime data (locations, dates, crime types, etc.)—enabling it to generate more accurate and relevant code.

\section{Qualitative Analysis}

We conducted one case study to illustrate the practical applicationa of our fine-tuned model in real-world scenarios. 

% The first case focuses on counterfactual evaluation of targeted interventions in crime hotspots, while the second explores geospatial queries related to crime data.

\subsection{Case: Counterfactual Evaluation of Targeted Interventions in Crime Hotspots of Guangdong Province}

Zhongshan, a city in Guangdong with around 4.4 million residents (2020), has implemented data-driven policing strategies. Official reports show crime reductions, with a 37\% drop in serious violence and 46\% in theft and robbery by mid-2020 \cite{Zhongshan2020}, highlighting the value of analyzing its crime data for policymaking.

To validate our fine-tuned LLM's capability as an analytical assistant for crime data interpretation, we designed three counterfactual scenarios that test the model's ability to process natural-language queries and generate actionable geospatial insights \cite{William2025}. Each scenario examines different dimensions of crime prevention and resource allocation strategies.



\noindent \textbf{Scenario 1: Night-time Crime Prevention Impact Analysis}

We posed the question: ``What percentage reduction in overall crime could be achieved if all night-time crimes in Guangdong Province during 2019 were prevented?'' This scenario evaluates the potential impact of enhanced nighttime patrol strategies or improved lighting infrastructure.

The model's analytical approach demonstrates sophisticated understanding of temporal crime patterns. As shown in Listing \ref{lst:q1}, the generated code correctly filters crimes by province and year, implements proper time-based categorization (defining night-time as 6 PM to 6 AM), and calculates the proportional impact of targeted interventions.

\lstinputlisting[style=pythoncode, label={lst:q1}]{codigos/q1.py}

\begin{tcolorbox}[colback=white!95!black, colframe=black, title=Q1 Response, fontupper=\scriptsize, boxsep=2pt, left=3pt, right=3pt]
  The province of Guangdong in China reported 1376 total crimes in 2019. Out of those, 633 crimes occurred at night. If all night-time crimes were prevented, the overall crime rate would decrease by 46.00\%. This is because 46.00\% of the total crimes in 2019 were night-time crimes.
\end{tcolorbox}

The analysis revealed that Guangdong Province recorded 1,376 total crimes in 2019, with 633 occurring during nighttime hours. The model concluded that preventing all night-time crimes would achieve a 46.00\% reduction in overall crime rates, indicating that nearly half of all criminal activity occurs during vulnerable nighttime periods.


\noindent \textbf{Scenario 1 - Questions Reformulation and Responses}

\begin{table}[hbtp]
\centering
\small
\begin{tabular}{|p{0.4\textwidth}|p{0.5\textwidth}|}
\hline
\textbf{Question} & \textbf{Response} \\
\hline
If all night-time crimes in Guangdong Province were completely prevented in 2019, what percentage drop in overall crime would result? & The province of Guangdong in China reported 1376 total crimes in 2019. If all nighttime crimes were eliminated, the crime rate would decrease by 46.00\%. This is because one of the outputs directly states that the percentage decrease in total crime if all nighttime crimes were eliminated is 46.00\%.. \\
\hline
What would be the estimated percentage decrease in total crime in Guangdong Province during 2019 if night-time crimes had been entirely eliminated? & The correct approach is to calculate the percentage of nighttime crimes in relation to total crimes and then use that percentage to estimate the decrease. In this case, the percentage of nighttime crimes in 2019 ranged from 31.57\% to 46.00\% across the different outputs.

Using the highest percentage (46.00\%), if all nighttime crimes were eradicated, the estimated percentage decrease in total crime would be 46.00\% because 46.00\% of the crimes in 2019 occurred at night. \\
\hline
How much could the overall crime rate in Guangdong Province for 2019 be reduced, in percentage terms, if all night-time crimes were prevented? & The correct total number of crimes in Guangdong Province for 2019 is 1376, with 957 daytime crimes and 419 nighttime crimes. If all night-time crimes were eliminated, the crime rate would decrease by 46.00\% (because the night-time crimes account for 30.45\% of the total crimes). \\
\hline
\end{tabular}
\caption{Scenario 1 Questions Reformulation and Responses}
\label{tab:questions_responses}
\end{table}

The model's responses consistently identify a 46\% reduction in total crime if nighttime incidents were eliminated, which aligns with the original data. However, subtle issues undermine reliability: Answer 1 uses vague phrasing (``one of the outputs'') that reduces clarity; Answer 2 introduces an unsupported ``31.57\% to 46\%'' range, creating confusion; and Answer 3 incorrectly reports nighttime crime as 419 instead of 633, leading to an internally inconsistent breakdown. These inconsistencies highlight the need for the model to provide more comprehensive explanations alongside numerical results, ensuring users can understand the reasoning process and verify calculations. The variations in supporting details across different query formulations demonstrate the model's current limitations in maintaining consistent reasoning when the same question is asked in different ways, suggesting a need for more diverse training examples that focus on detailed crime temporal analysis.


\noindent \textbf{Scenario 2: Spatial Crime Diffusion Effects}  % TODO: mejorar

The second scenario examined spatial crime relationships: ``How would the total number of crimes across all counties within a 70 km radius of the point [22.516998, 113.392723] change if crime in that area were reduced by 20\%?'' This question tests the model's capacity for geospatial reasoning and understanding of crime diffusion effects.

The model's implementation, presented in Listing \ref{lst:q2}, demonstrates interesting geospatial processing capabilities. It correctly creates a buffer zone around the specified coordinates, identifies neighboring counties through geometric intersection analysis, and simulates crime reduction scenarios through statistical sampling.

\lstinputlisting[style=pythoncode, label={lst:q2}]{codigos/q2.py}

\begin{tcolorbox}[colback=white!95!black, colframe=black, title=Q2 Response, fontupper=\scriptsize, boxsep=2pt, left=3pt, right=3pt]
  The current total number of crimes in the area is 68,247. If crime in that area were reduced by 20\%, the new total would be approximately 54,598. This means about 13,649 crimes would be reduced.
\end{tcolorbox}

This result indicates a direct spatial spillover effect: a 20\% reduction in crime at the central location leads to an equivalent 20\% decrease in total crimes across the surrounding area. Specifically, the model predicts a drop from 68,247 incidents to 54,598, a reduction of 13,649 crimes, demonstrating a proportional relationship between crime reduction at the focal point and its adjacent counties.

\noindent \textbf{Scenario 2 - Questions Reformulation and Responses}
\begin{table}[hbtp]
\centering
\small
\begin{tabular}{|p{0.4\textwidth}|p{0.5\textwidth}|}
\hline
\textbf{Question} & \textbf{Response} \\
\hline
What impact would a 20\% reduction in crime at [22.516998, 113.392723] have on the combined crime count for all counties within a 70 km radius? & The current total crime count in the 70 km radius is 68,247. If there's a 20\% reduction in crime, the new total crime count would be 54,598. This reduction would affect all 331 counties within the 70 km radius. \\
\hline
 By how much would the total crimes across all counties within 70 km of [22.516998, 113.392723] decrease if crime in that zone dropped by 20\%? & Not possible to answer. \\
\hline
What would be the decrease in overall crime across all counties within 70 km of [22.516998, 113.392723] if crime at that central point fell by 20\%? & 
The crime count within 70 km of the specified location is currently 68,247. If crime at that central point decreased by 20\%, it would drop to 54,598. This means there would be a 20\% decrease in crime.

This is because the current average crime rate of 206.18 crimes per county decreased by 20.00\%. \\
\hline
\end{tabular}
\caption{Scenario 2 Questions Reformulation and Responses}
\label{tab:questions_responses_2}
\end{table}

The review of the responses shows that the initial count of 68,247 crimes within a 70 km radius of [22.516998, 113.392723], when a 20\% reduction is applied at the central point, becomes 54,598 crimes (13,649 fewer incidents). This calculation remains consistent across the reformulated questions, except for the only one case in which the model was unable to make a coherent answer.


\noindent \textbf{Scenario 3: Reporting Rate Impact on Safety Rankings}

The final scenario explored data quality implications: ``Considering the top 10 safest counties in Guangdong Province, how would this ranking change if crime reporting increased by 15\% in Zhongshan City in 2019?'' This question examines how improved reporting mechanisms might alter safety assessments.

The model's approach, illustrated in Listing \ref{lst:q3}, systematically ranks counties by crime frequency, simulates increased reporting rates, and evaluates ranking changes. The analysis revealed that despite a 15\% increase in reported crimes (from 80 to 92 incidents), Zhongshan City maintained its fourth position among the safest counties.

\lstinputlisting[style=pythoncode, label={lst:q3}]{codigos/q3.py}

\begin{tcolorbox}[colback=white!95!black, colframe=black, title=Q3 Response, fontupper=\scriptsize, boxsep=2pt, left=3pt, right=3pt]
  Based on the provided code outputs, the current top 10 safest counties in Guangdong Province for 2019 are:

  1. Bao'an District with 298 crimes

  2. Dongguan City with 139 crimes
  
  3. Baiyun District with 113 crimes
  
  4. Zhongshan City with 80 crimes
  
  5. Longgang District with 78 crimes
  
  6. Panyu District with 62 crimes
  
  7. Shunde District with 61 crimes
  
  8. Chancheng District with 56 crimes
  
  9. Nanshan District with 55 crimes
  
  10. Haizhu District with 49 crimes

  If the crime reporting in Zhongshan City increased by 15\% in 2019, the new top 10 safest counties would be:

  1. Bao'an District with 298 crimes

  2. Dongguan City with 139 crimes

  3. Baiyun District with 113 crimes

  4. Zhongshan City with 92 crimes (after 15\% increase)

  5. Longgang District with 78 crimes

  6. Panyu District with 62 crimes

  7. Shunde District with 61 crimes

  8. Chancheng District with 56 crimes

  9. Nanshan District with 55 crimes

  10. Haizhu District with 49 crimes

  The change in ranking is due to Zhongshan City moving from 4th to 4th place after the increase in crimes.
\end{tcolorbox}


\noindent \textbf{Scenario 3 - Questions Reformulation and Responses}

\begin{table}[hbtp]
\centering
\small
\begin{tabular}{|p{0.4\textwidth}|p{0.5\textwidth}|}
\hline
\textbf{Question} & \textbf{Response} \\
\hline
How would a 15\% increase in crime reporting in Zhongshan City during 2019 impact the rankings of the top 10 safest counties in Guangdong Province? & Zhongshan City would remain in 4th place despite the increase in reported crimes. \\
\hline
If crime reporting in Zhongshan City had increased by 15\% in 2019, what impact would that have on the ranking of the top 10 safest counties in Guangdong Province? &The code calculates the adjusted crime counts for each county with a 15\% increase but encounters errors when trying to analyze Zhongshan City's data (which is 0 crimes in all cases). As a result, the adjusted rankings remain identical to the original rankings.. \\
\hline
In the context of the top 10 safest counties in Guangdong Province, what changes would occur in the rankings if Zhongshan City's crime reports rose by 15\% in 2019? & Not possible to answer. \\
\hline
\end{tabular}
\caption{Scenario 3 Questions Reformulation and Responses}
\label{tab:questions_responses_3}
\end{table}

The model's Scenario 3 responses exhibit varied relevance and task completion: only the first answer correctly states that Zhongshan would stay fourth after a 15\% crime increase, directly addressing the ranking impact, while the second and third responses either fabricate data errors or refuse to answer, thus failing task completion.


\noindent \textbf{Analytical Implications and Future Potential}

These case studies demonstrate the model's practical utility despite the quantitative metrics presented in Table \ref{tab:combined_metrics}. While our fine-tuned Llama3 model shows room for improvement in compilation success rates, the generated code consistently exhibits interesting capabilities in data filtering, temporal analysis, geospatial processing, and statistical simulation, all essential for crime data analysis.

The qualitative analysis reveals that even with current limitations, the model successfully transforms complex criminological questions into executable analytical workflows with contextually appropriate interpretations. This bridges the gap between technical analysis and policy-relevant insights, enabling law enforcement agencies to make informed decisions about resource allocation and intervention strategies.

These promising results suggest that with continued refinement, particularly through expanded domain-specific datasets and targeted improvements to reduce compilation errors, our approach could become a valuable tool for evidence-based policing initiatives. The counterfactual analysis framework validates the potential of fine-tuned LLMs for supporting data-driven decision-making in public safety contexts.


\section{Final considerations}

This chapter evaluated our fine-tuned Llama3-8B-Instruct model for crime data analysis, demonstrating that despite higher compilation error rates compared to o4-mini (Percent Error@k: 0.233 vs 0.003), the model significantly outperformed in solution accuracy (Pass@k: 0.742 vs 0.303), indicating superior semantic understanding of crime-related queries. Through three scenarios analyzing Guangdong Province crime data, we validated the model's practical utility for evidence-based policing, revealing that night-time crime prevention could reduce overall crime by 46\%, spatial interventions produce significant spillover effects (13 649 crime reduction from a 20\% reduction at a central point), and safety rankings remain stable despite improved reporting rates. The results confirm that fine-tuned LLMs can effectively bridge technical data analysis with policy-relevant insights, though expanding the domain-specific dataset remains crucial for reducing compilation errors and enhancing generalization across varied crime patterns.


% \customchapter{CONCLUSIONS AND FUTURE WORK} 

\section{Conclusions}


% Este trabajo es el primero buscar sentar la base para el uso de modelos de lenguaje grande (LLMs) en la evaluacion de data criminal. El enfoque de usar una LLM finetuneada demuestra que se requiere un dataset mas robusto especializado para que el modelo pueda analizar correctamente los datos baseado en generacioon de codigo como se hace para el ambio de resolucion de problemas de matematica.
% Usando un enfoque de synthetic data generation usando un modelo mas grande, demostramos a traves de un caso de estudio que las LLMs tienen el potencial de ser una herramienta valiosa para la evaluacion de datos criminales, permitiendo a los usuarios realizar consultas complejas y obtener respuestas trazables.
% This study ... By using ... These findings ...

This study represents the first comprehensive approach to establish a foundation for using large language models (LLMs) in criminal data evaluation through code-backed answers. By using a fine-tuned LLaMA3-8B model trained on the ChinaCrimeQACode dataset containing 5,000 question-code pairs, we highlight the necessity of a more robust, domain-specific dataset for the model to accurately analyze crime data. These findings reveal that while specialized datasets are essential for effective crime data analysis, the approach of code generation, similar to mathematical problem-solving domains, enables traceable and interpretable responses that can facilitate informed decision-making for both citizens and law enforcement authorities.

% The approach of synthetic data generation using a larger model demonstrates that LLMs have the potential to be valuable tools for crime data evaluation, enabling users to perform complex queries and obtain traceable answers. 


\section{Limitations and Future Work}

During our experiments with the finetuned model, we identified several significant limitations in both the model and the dataset. Addressing these limitations is essential for enhancing the model's performance and usability. For example, when queries include coordinates (latitude and longitude) in a format different from the dataset's standard ('[lat, long]'), the model consistently fails to provide accurate or approximate answers. Additionally, the model struggles with queries that contain misspellings or alternative names for specific information, such as locations or types of crimes, that differ from those in the dataset. For instance, when asked "How many crimes occurred in Guangdong?" instead of "How many crimes occurred in Guangdong Province?", the model is unable to produce a correct response.

To address these challenges, future work will focus on enhancing both the dataset and the model. For the dataset, we propose expanding its geographical coverage to include regions such as Brazil, New York, and potentially Lima, thereby improving the model's generalizability across diverse contexts. Additionally, incorporating a "LLM-as-a-judge" mechanism during the dataset construction phase, by verifying whether the model's outputs for paraphrased or altered questions align with the original, could enhance label reliability and consistency. Scaling the dataset to at least 10,000 records is another priority, as this would provide a richer training environment and enable the model to generalize more effectively across a broader range of queries.

On the modeling side, we aim to enhance the alignment between the model’s textual outputs and the visual feedback provided to users. This can be achieved by integrating a visualization tool capable of displaying maps and landmarks related to the model's predictions, offering users a more intuitive understanding of the model’s reasoning process. Additionally, incorporating the model into an agentic pipeline, where it collaborates with external tools or agents, could significantly improve its ability to handle complex, multi-step queries in real-world applications. Furthermore, introducing a preprocessing step before the model generates responses, where a smaller model identifies the key points of the question and refines it by replacing these key points with corresponding values from the dataset, could enhance the model's accuracy and relevance in addressing user queries.

%\item Reducing the token usage requesting only the function solve and not the full import of the libraries.
%\section{Limitations and Future Work}

%During our experiments with the finetuned model, we identified several significant limitations in both the model and the dataset. Addressing these limitations is essential for enhancing the model's performance and usability. For example, when queries include coordinates (latitude and longitude) in a format different from the dataset's standard ('[lat, long]'), the model consistently fails to provide accurate or approximate answers. Additionally, the model struggles with queries that contain misspellings or alternative names for specific information, such as locations or types of crimes, that differ from those in the dataset. For instance, when asked "How many crimes occurred in Guangdong?" instead of "How many crimes occurred in Guangdong Province?", the model is unable to produce a correct response.

%\begin{itemize}
 %   \item \textbf{Dataset Improvements:} 
  %  \begin{itemize}
   %     \item Expand the range area to Brazil and NY (and possibly Lima). 
    %    \item Use the LLM-as-a-judge in the dataset construction phase.
     %   \item Scale the dataset to 10K records.
    %\end{itemize}
    %\item \textbf{Model Improvements:}
    %\begin{itemize}
     %   \item Relate the output of the model with the visualization feedback.
     %   \item Use the model as a part of Agentic pipeline
    %\end{itemize}
%\end{itemize}
%\input{secciones/recomendaciones} % sección opcional


%% ============================================================================
\renewcommand{\bibname}{\hfill\Large\bf{BIBLIOGRAPHY}\hfill}

\bibliographystyle{IEEEtran} % Estableciendo el estilo de citas IEEEtram.

\bibliography{referencias} % Recibe las referencias de IEEE

% \chapter*{\center \Large APPENDIX} 
\addcontentsline{toc}{section}{\bfseries APPENDIX} 
\markboth{APPENDIX}{APPENDIX} 

% \par Los algoritmos desarrollados .....
\appendix
\renewcommand{\thesection}{A.\arabic{section}}


\section{Dataset specification}
\label{appendix:dataset}



\begin{table}[H]
\centering
\caption{Dataset Description: \texttt{crimes\_df} – Crime Records}
\begin{tabular}{|l|l|p{8cm}|}
\hline
\textbf{Column} & \textbf{Type} & \textbf{Description} \\
\hline
case\_type & string & Category of the crime incident (e.g., theft, assault). \\
latitude & float & Latitude of the incident location. \\
longitude & float & Longitude of the incident location. \\
incident\_location & string & Textual description of the incident location. \\
incident\_province & string & Province where the incident occurred. \\
incident\_city & string & City where the incident occurred. \\
incident\_county & string & County or district of the incident. \\
formatted\_datetime & datetime & Standardized timestamp of the incident. \\
street\_name & string & Name of the street where the incident occurred. \\
geometry & geospatial point & Geographic point of the incident. \\
\hline
\end{tabular}
\end{table}


\begin{table}[H]
\centering
\caption{Dataset Description: \texttt{streets\_df} – Street Network}
\begin{tabular}{|l|l|p{8cm}|}
\hline
\textbf{Column} & \textbf{Type} & \textbf{Description} \\
\hline
street\_name & string & Official name of the street (can be linked to \texttt{crimes\_df.street\_name}). \\
geometry & line polygon & Geometric representation of the street. \\
incident\_province & string & Province where the street is located. \\
\hline
\end{tabular}
\end{table}


\begin{table}[H]
\centering
\caption{Dataset Description: \texttt{geometries\_df} – Administrative Boundaries}
\begin{tabular}{|l|l|p{8cm}|}
\hline
\textbf{Column} & \textbf{Type} & \textbf{Description} \\
\hline
name & string & Name of the administrative region. \\
geometry & polygon & Boundary geometry of the region. \\
geom\_type & string & Administrative level (e.g., province, city, county). \\
\hline
\end{tabular}
\end{table}


% Later in your document, you can reference this appendix with:
% As shown in Appendix~\ref{appendix:dataset}, the dataset...

\section{Prompts}
\label{appendix:prompts}


\begin{lstlisting}[language=txt, caption={Prompt for generating Python code solutions}, label={lst:prompt_code_generation}, basicstyle=\ttfamily\small, breaklines=true, columns=fullflexible, keepspaces=true]

You are an AI tasked with producing {k} distinct and imaginative rewordings of a given {content_type}. Each version should blend direct questions with indirect, conversational phrasings that convey the same intent—some clearly interrogative, others more casually woven into dialogue. Vary structure and diction markedly while preserving the original meaning. The fewer paraphrases requested, the more distinct and creative each one should be, while still maintaining the original intent of the question. Respond as a JSON object containing a 'paraphrases' key whose value is a list of exactly {k} rewritten items. Re‑order arguments when necessary, since the prompts are template‑generated. Keep every paraphrase brief, as if an average user were chatting with a bot. Any terms wrapped in <> must remain unchanged. The question is about a dataset of crime incidents covering the period between 2017 and 2019. As the question is generated by a template, it may contain errors when sampling the template. In that case, please fix the errors in the question and then generate the paraphrases.
\end{lstlisting}

\end{document}  