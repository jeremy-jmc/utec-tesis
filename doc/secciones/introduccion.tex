\chapter{INTRODUCTION} 
% ¿Cuál es el contexto del tema? Contextualice al lector, inicia con un párrafo que le genere interés por el tema.
% ¿Cuál es el problema?
% ¿Qué han hecho otros para resolver el problema?
% ¿Qué propone el artículo?
% ¿En qué se diferencia tu trabajo del resto? (Cuál es el aporte)


% \section{Presentación del tema de investigación}   % Antecedentes, problematica y justificacion
\section{Introduction to the Research Topic} % TODO: Focus on social impact and problematic

% Data visualization is a powerful tool for data explofration and insight communication \cite{Liu2024NLDriven}.

% \section{Descripción de la situación problemática} % Planteamiento del problema
\section{Description of the Problematic Situation}

% * Paragraph 1 – General Context
In many large cities, concerns about crime have become increasingly common, especially in areas frequently visited by tourists \cite{SopikoTevdoradze2024CrimeTourism}. Popular destinations are often associated with considerable levels of criminality, which can discourage visitors from exploring freely \cite{Machado2012CrimeRJ}. However, the impact of urban crime extends beyond tourists. Local residents, while more familiar with their surroundings, typically rely on personal experience, local news, and community word-of-mouth to decide which streets or neighborhoods to avoid. This informal awareness helps them navigate perceived danger zones, but it does not fully shield them from the risks that exist in their own cities.

% Many popular tourist destinations are often associated with considerable crime rates, which can deter visitors and affect their willingness to explore these areas. Besides, big cities like New York and Chicago 
% For instance, cities like New York, Los Angeles, and Chicago have been known to experience significant crime rates, particularly in certain neighborhoods. This can lead to a perception of danger among tourists and residents alike, ultimately affecting their willingness to explore these areas.  negatively impact local economies.

% * Paragraph 2 – Impact and Limitations of Current Systems
In response to these concerns, several visualization tools (\cite{Garcia2022CriPAV}, \cite{Salah2022BigCDVis}, \cite{Silva2017CrimeVisAI}, \cite{Garcia2020MiranteAV}, \cite{Garcia2021CrimAnalyzer}) have emerged to increase awareness, validate hypotheses, and support safer decision-making. However, these solutions are often difficult to use for non-expert users. They usually provide sophisticated analysis using robust feature engineering processes, restricting their accessibility to a broader audience. As a result, both citizens and authorities are left with insufficient support to anticipate risks or act proactively.

% * Paragraph 3 – Analytical Challenges
A key challenge in crime prevention is the lack of accessible tools that help both citizens and authorities interpret complex crime data. While large volumes of data are publicly available (\cite{Zhang2025CrimeDatasetChina}, \cite{NYCDataset}, \cite{ChicagoDataset}), they're often difficult to analyze without technical skills or time \cite{Zengli2020CrimePatterns}. Crime patterns constantly shift across time and space. Some areas may be riskier at night, others during weekends or certain seasons. Without tools that clearly communicate these dynamics in real time, people struggle to make safe decisions, and authorities face obstacles in allocating resources effectively.

% * Paragraph 4 – Gaps in SOTA Methods
Although recent tools have improved the visual representation of crime data, they still present critical limitations in usability and applicability. Many platforms depend heavily on user expertise and offer mostly static or pre-defined visualizations that demand manual interpretation \cite{Garcia2022CriPAV}. This limits their value in dynamic, real-world scenarios where fast decisions are required. Without systems that deliver clear, real-time insights tailored to a user's context, both citizens and authorities remain underserved in their efforts to understand and respond to urban crime.

% TODO: Users may lack the knowledge or skills to interact with the visualization effectively. Without proper guidance or intuitive interaction design, users may struggle to fully leverage the interactive capabilities of the visualization. \cite{Liu2024NLDriven}

% \section{Formulación del problema} % Planteamiento del problema
% \section{Problem Statement} % Statement of the problem


% \section{Justificación}    % Antecedentes, problematica y justificacion
\section{Justification}

% TODO: ! Paragraph 1: Context and Urgency


% ! Paragraph 2: Enabling Technology
Large language models (LLMs) offer a human-centred interface for interacting with complex data \cite{Yang2024HumanAIInteraction} \cite{Pappula2023LLMsFC}. Unlike, web dashboards, GIS software or specialized tools, LLM-powered chatbot lets users ask questions like ``Which streets saw the largest surge in robberies in the last month?'' and instantly receive concise explanations. In addition, open-source models alleviate many of the privacy concerns often associated with proprietary systems and can be more cost-effective \cite{Liu2024NLDriven}. 
% Many research like \cite{Liu2024NLDriven} uses LLMs to assist in dataset constructions and training small LLMs, because it can be more cost-effective and data-privacy friendly.

% LLMs have significantly enhanced chatbots and virtual assistants, enabling more contextually aware and human-like interactions across various domains 
% With the advent of LLMs, there has been a growing in the interaction via chat to several tasks, like web-browsing, text summarization, question-answering, code generation, among others.

% ! Paragraph 3: Rationale for an LLM Layer 
Adopting an LLM-based conversational layer directly addresses the two principal weaknesses of current crime-analytics platforms: accessibility and timeliness. First, a chat interface removes the steep learning curves of existing tools. Second, by supporting rapid, iterative exploration of temporal windows, geographic areas and crime types, it delivers immediate insights when users need them.

% ! Paragraph 4: Anticipated Impact & Wider Significance 
In this study, we aim to enhance crime-prevention efforts and raise awareness about the risks associated with routes used by tourists and residents by developing an intuitive chat interface. This interface will allow users to interact with crime data, presenting both textual summaries and dynamic visual feedback, to facilitate informed decision-making. Additionally, the work establishes a foundation for future investigations into the application of LLMs to more sophisticated crime-analysis tasks. % , such as prediction, topological analysis, and clustering | leveraging in the CoT process

% \section{Objetivos de investigación}   % Objetivos
% \section{Scope and Limitations / Constraints} % Scope and expected benefits
% \section{Alcance y limitaciones / restricciones} % Alcance y beneficios esperados
\section{Research Objectives} % Objectives

\subsection{General Objectives}
To develop an intuitive and accessible system that leverages LLMs to enhance crime prevention efforts by enabling users to interact with spatio-temporal crime data through natural language queries, thereby improving decision-making for both citizens and authorities.

\subsection{Specific Objectives}
\begin{itemize}
    \item Design and implement a chat-based interface that allows users to query crime data and receive both textual and visual feedback.
    \item Address privacy and security concerns by focusing on the use of open-source LLMs as alternative solutions to proprietary models.
    \item Implement and agent-based system that utilizes LLMs to process and analyze spatio-temporal crime data, providing real-time insights and recommendations.
    % \item Ensure the system is user-friendly and accessible to non-expert users, minimizing the need for technical expertise.
    % \item Lay the groundwork for future research on advanced LLM-based analyses, such as prediction, topological analysis, and clustering.
\end{itemize}

