\customchapter{ABSTRACT} 

% \begin{center}
% \large \vspace{-1.5cm} \textbf{A CONVERSATIONAL LLM-POWERED SYSTEM FOR SPATIO-TEMPORAL CRIME DATA ANALYSIS}
% \end{center}

Large Language Models (LLMs) have demonstrated exceptional versatility across diverse domains, yet their application in crime analytics remains underexplored due to a lack of domain-specific datasets and specialized methodologies. To address this gap, we introduce ChinaCrimesQACode, a novel dataset designed to capture the intricacies of spatio-temporal crime analysis, including detailed crime incidents, street networks, and administrative boundaries with location-specific queries. Leveraging ChinaCrimesQACode, we focus on code generation from natural language questions, enabling LLMs to produce high-quality, executable Python code for geospatial crime data analysis from structured conversational inputs. Fine-tuned model and baseline models were evaluated using semantic equivalence metrics and LLM-as-a-judge. Our results demonstrate substantial improvements in generating contextually accurate analytical code, highlighting the transformative potential of tailored datasets and fine-tuning methodologies in optimizing crime intelligence workflows. This work highlights the potential of LLMs in crime analytics workflows and the essential role of domain-specific datasets in tailoring them to specialized analytical challenges.

\noindent \textbf{Keywords:}\\
\noindent Large Language Models; Crime Analytics; Geospatial Analysis; Code Generation; Conversational AI