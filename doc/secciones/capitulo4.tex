\chapter{ EXPERIMENTS AND RESULTS}

% \section{Protocolo de entrenamiento}
% \subsection{Protocolo de experimentación}

\section{Experiments}

\section{Results}

\begin{table}[h!]
\centering
\caption{Evaluation Metrics: Mean, Std, and Median}
% \resizebox{\textwidth}{!}
{%
\begin{tabular}{l|ccc}
\toprule
\textbf{Metric} & \textbf{Mean} & \textbf{Std} & \textbf{Median} \\
\midrule
perc\_error@ & 0.233 & 0.299 & 0.125 \\
pass@k    & 0.742 & 0.437 & 1.0   \\
pass\_over\_k  & 0.460 & 0.410 & 0.406 \\
code\_bleu@k  & 0.355 & 0.044 & 0.353 \\
\bottomrule
\end{tabular}
}
\label{tab:metrics_generated}
\end{table}

\section{Discussion}

\section{Case studies}

\subsection{Case I: Counterfactual Evaluation of Targeted Interventions in High-Crime Zones of Suzhou (Jiangsu)}

\begin{itemize}
    \item \textbf{Case summary:} This analysis examines the hypothetical effects of intensive police interventions targeting the most crime-prone streets in Suzhou, Jiangsu Province. The study is motivated by the growing interest in situational crime prevention strategies \cite{clarke1995situational}, which propose that altering or enhancing surveillance in urban areas can influence criminal behavior. 
        \item \textbf{General Objective:} To evaluate the potential impact of securing the five most dangerous streets in Suzhou on the city’s overall crime rates, using data collected between 2020 and 2022.
        \item \textbf{Justification:} The counterfactual methodology provides insights into a critical public policy question: “What would be the outcome if criminal activities in key urban hotspots were entirely prevented?” Leveraging generative AI for this estimation aids in prioritizing security resources and assessing the cost-effectiveness of targeted interventions without requiring real-world implementation. Additionally, Suzhou’s status as a highly developed city in Jiangsu makes it an ideal setting for studying predictive-preventive urban strategies.
        \item \textbf{Proposed questions:}
        \begin{itemize}
            \item "How would securing the top 5 most dangerous streets in Suzhou affect the city’s overall crime rates?"
            \item "What percentage reduction in overall crime could be achieved if all night-time thefts in Suzhou during 2021 were prevented?"
            \item "If theft crime-category decreased by 10\% annually, in which year would Suzhou city record fewer than 500 such crimes?"
            \item "If crime reporting increased by 15\% in Suzhou city, how would that affect the city's safety ranking in Jiangsu province?"
        \end{itemize}




\end{itemize}


\subsection{Case II:}

\subsection{Case III:}

% \chapter{MARCO METODOLÓGICO}

% Inicie aquí el texto, utilizando sangría de 1.25 cm. en el primer
% párrafo. Continúe el segundo párrafo.

% Continúe el segundo párrafo

% \section{Primer subtítulo}

% En la actualidad, las unidades de información hacen frente a muchos cambios
% debido a los avances tecnológicos, explosión informativa, nuevos recursos y
% soportes, por lo cual se implementan servicios innovadores que les permitan a
% sus usuarios tener acceso a muchas fuentes de información. Para asegurar el
% acceso y uso de los servicios los usuarios requieren poseer una serie de
% habilidades que les permitan identificar, recuperar, manejar, discernir,
% organizar, utilizar y comunicar la información de manera eficaz para la toma de
% decisiones.

% Entre las causas se pueden considerar la falta de tiempo asignado al taller,
% las limitaciones en cuanto a laboratorios, el poco personal, la falta de una
% correcta selección de los contenidos, así como, una calificación que asegure el
% cumplimiento de los objetivos.

