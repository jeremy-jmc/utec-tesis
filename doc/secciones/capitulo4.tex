\lstdefinestyle{pythoncode}{
  language=Python,
  basicstyle=\scriptsize\ttfamily,
  backgroundcolor=\color{gray!5},
  keywordstyle=\color{blue},
  commentstyle=\color{green!50!black},
  stringstyle=\color{orange},
  numberstyle=\tiny\color{gray},
  numbers=left,
  stepnumber=1,
  numbersep=8pt,
  frame=single,
  breaklines=true,
  captionpos=b,
  showstringspaces=false,
  tabsize=4
}
% Ajuste opcional del formato de captions
\captionsetup[lstlisting]{font=small,labelfont=bf}



\chapter{ EXPERIMENTS AND RESULTS}

This chapter presents the outcomes derived from applying the methodology detailed in the preceding chapter. We begin by describing the experimental setup and specifying the hyperparameters employed. Next, we report the results from the evaluated models, covering both quantitative and qualitative analyses. Lastly, we discuss the significance of these findings and include case studies to demonstrate the practical relevance of our results.

\section{Experimental setup}

\section{Hyperparameter selection}

\section{Quantitative Results}

% TODO: TABLE 1 comparing Phi3-mini4k-instruct, Llama3-8B-Instruct, baseline and trained models along with Closed-Source models like GPT-4o-models, pass@k with k = 1 (with greedy decoding), 4, 16 (multinomial sampling)

\begin{table}[h!]
\centering
\caption{Evaluation Metrics finetuned LLaMA3: Mean, Std, and Median (Test Set)}
% \resizebox{\textwidth}{!}
{%
\begin{tabular}{l|ccc}
\toprule
\textbf{Metric} & \textbf{Mean} & \textbf{Std} & \textbf{Median} \\
\midrule
perc\_error@ & 0.233 & 0.299 & 0.125 \\
pass@k    & 0.742 & 0.437 & 1.0   \\
pass\_over\_k  & 0.460 & 0.410 & 0.406 \\
code\_bleu@k  & 0.355 & 0.044 & 0.353 \\
\bottomrule
\end{tabular}
}
\label{tab:metrics_generated}
\end{table}

\begin{table}[h!]
\centering
\caption{Evaluation Metrics for o4-mini: Mean, Std, and Median (Test Set)}
{%
\begin{tabular}{l|ccc}
\toprule
\textbf{Metric} & \textbf{Mean} & \textbf{Std} & \textbf{Median} \\
\midrule
perc\_error@k     & 0.003 & 0.014 & 0.0   \\
pass\_at@k        & 0.303 & 0.459 & 0.0   \\
pass\_over\_k      & 0.063 & 0.175 & 0.0   \\
code\_bleu@k      & 0.370 & 0.054 & 0.375 \\
\bottomrule
\end{tabular}
}
\label{tab:metrics_new_data}
\end{table}


\section{Discussion}

Based on the observed metrics, several conclusions emerge regarding the performance of the fine-tuned LLaMA3‑8B‑Instruct model versus the base o4‑mini. First, LLaMA3‑8B‑Instruct exhibits a higher perc\_error@k than o4‑mini; yet it significantly outperforms o4‑mini in both pass@k and pass\_over\_k. This indicates that, despite a slightly lower rate of successful compilation, the fine‑tuned model produces more precise and semantically relevant solutions. In contrast, o4‑mini’s better perc\_error@k can be misleading: it often compiles code that runs but fails to solve the problem correctly.

This discrepancy is likely rooted in architectural and training differences. o4‑mini is believed to use a Mixture‑of‑Experts (MoE) setup with approximately 40B total parameters—only about 8B of which are active per inference—whereas LLaMA3‑8B‑Instruct is a dense model with all 8B parameters actively contributing . While MoE architectures provide access to a broader parameter pool, this capacity advantage does not automatically translate to superior domain-specific performance. In contrast, the fully fine‑tuned LLaMA model has internalized domain-specific patterns—such as those needed to manipulate crime data (locations, dates, crime types, etc.)—enabling it to generate more accurate and relevant code.
\section{Qualitative Analysis}

% \section{Discussion}

\section{Case studies}

\subsection{Case I: Counterfactual Evaluation of Targeted Interventions in Crime Hotspots of Guangdong Province}

\begin{itemize}
       \item \textbf{Case summary:} This study models what might happen if targeted interventions (e.g., CCTV, policing) in crime hotspot areas of Zhongshan (Guangdong Province) completely prevented certain crimes in 2019. Drawing on situational crime prevention literature, especially meta-analyses showing modest but significant crime reductions from CCTV, typically 16–20\% in property and vehicle crime settings, this case uses counterfactual modeling to simulate those interventions in Zhongshan.
      \item \textbf{General Objective:} To simulate the following counterfactual scenarios using 2019 data:
      \begin{enumerate}
          \item Entire prevention of night-time thefts in Zhongshan and its effect on total crime.
          \item A 20\% reduction in total crime at the central hotspot ([22.516998, 113.392723]) and its spatial impact on average crime in counties within a 70 km radius.
          \item A 15\% increase in reported crime in Zhongshan and how that would alter the Top 10 safest county rankings in Guangdong Province.
      \end{enumerate}
      \item \textbf{Justification:} Meta-analyses consistently find CCTV leads to modest, significant reductions—around 16\% overall and ~20\% for property/vehicle crimes, and are more effective when actively monitored or combined with other interventions. Zhongshan, as a rapidly urbanizing city in Guangdong, offers a relevant setting to assess how targeted counterfactuals can inform policy on hotspot interventions and reporting practices.

        \item \textbf{Proposed questions:}
        \begin{itemize}
            \item "What percentage reduction in overall crime could be achieved if all night-time thefts in Suzhou during 2019 were prevented?"
            \item "How would the average crime in neighboring counties (within a 70 km radius of [22.516998, 113.392723]) change if crime at that central point were reduced by 20\%?"
            \item "Considering the top 10 of the safest all counties in Guangdong Province, How would this top 10 change if the crime reporting increased by 15\% in Zhongshan City in 2019?"
        \end{itemize}
\end{itemize}

\subsubsection{Case I.1:}
\begin{itemize}
    \item \textbf{Question:} "What percentage reduction in overall crime could be achieved if all night-time thefts in Guangdong Province during 2019 were prevented?"
    \item \textbf{Model Code Generation:} One of the six code generations that were made by the model is shown below. 

    \lstinputlisting[
    style=pythoncode,    
    label={lst:q1}
    ]{codigos/q1.py}



    \item \textbf{Model Response:} 
    
    \begin{tcolorbox}[colback=white!95!black, colframe=black, title=Q1 Response, fontupper=\scriptsize]

    "The province of Guangdong in China reported 1376 total crimes in 2019. Out of those, 633 crimes occurred at night. If all night-time crimes were prevented, the overall crime rate would decrease by 46.00\%. This is because 46.00\% of the total crimes in 2019 were night-time crimes."
    \end{tcolorbox}

\subsubsection{Case I.2:}
\begin{itemize}
    \item \textbf{Question:} "How would the average crime in neighboring counties (within a 70 km radius of [22.516998, 113.392723]) change if crime at that central point were reduced by 20\%?"
    \item \textbf{Model Code Generation:} One of the six code generations that were made by the model is shown below. 


    \lstinputlisting[
    style=pythoncode,    
    label={lst:q2}
    ]{codigos/q2.py}

  \item \textbf{Model Response:} 
    

    \begin{tcolorbox}[colback=white!95!black, colframe=black, title=Q2 Response, fontupper=\scriptsize]
    
    The average crime rate in neighboring counties would decrease by 13.85\% if crime at the central point were reduced by 20\%. 

    This is because the simulated reduced crime rate in neighboring counties decreased from 4874.79 to 4199.85.
    
    \end{tcolorbox}

\end{itemize}
\subsubsection{Case I.3:}
\begin{itemize}
    \item \textbf{Question:} "Considering the top 10 of the safest all counties in Guangdong Province, How would this top 10 change if the crime reporting increased by 15\% in Zhongshan City in 2019?"
    \item \textbf{Model Code Generation:} One of the six code generations that were made by the model is shown below.

    \lstinputlisting[
    style=pythoncode,    
    label={lst:q3}
    ]{codigos/q3.py}
    
    
    \item \textbf{Model Response:} 
\begin{tcolorbox}[colback=white!95!black, colframe=black, title=Q3 Response, fontupper=\scriptsize]
Based on the provided code outputs, the current top 10 safest counties in Guangdong Province for 2019 are:

\begin{enumerate}
  \item Bao'an District with 298 crimes
  \item Dongguan City with 139 crimes
  \item Baiyun District with 113 crimes
  \item Zhongshan City with 80 crimes
  \item Longgang District with 78 crimes
  \item Panyu District with 62 crimes
  \item Shunde District with 61 crimes
  \item Chancheng District with 56 crimes
  \item Nanshan District with 55 crimes
  \item Haizhu District with 49 crimes
\end{enumerate}

If the crime reporting in Zhongshan City increased by 15\% in 2019, the new top 10 safest counties would be:

\begin{enumerate}
  \item Bao'an District with 298 crimes
  \item Dongguan City with 139 crimes
  \item Baiyun District with 113 crimes
  \item Zhongshan City with 92 crimes (after 15\% increase)
  \item Longgang District with 78 crimes
  \item Panyu District with 62 crimes
  \item Shunde District with 61 crimes
  \item Chancheng District with 56 crimes
  \item Nanshan District with 55 crimes
  \item Haizhu District with 49 crimes
\end{enumerate}

The change in ranking is due to Zhongshan City moving from 4th to 4th place after the increase in crimes.
\end{tcolorbox}

    
\end{itemize}

\subsection{Case II: Geospatial Queries}


% \chapter{MARCO METODOLÓGICO}

% Inicie aquí el texto, utilizando sangría de 1.25 cm. en el primer
% párrafo. Continúe el segundo párrafo.

% Continúe el segundo párrafo

% \section{Primer subtítulo}

% En la actualidad, las unidades de información hacen frente a muchos cambios
% debido a los avances tecnológicos, explosión informativa, nuevos recursos y
% soportes, por lo cual se implementan servicios innovadores que les permitan a
% sus usuarios tener acceso a muchas fuentes de información. Para asegurar el
% acceso y uso de los servicios los usuarios requieren poseer una serie de
% habilidades que les permitan identificar, recuperar, manejar, discernir,
% organizar, utilizar y comunicar la información de manera eficaz para la toma de
% decisiones.

% Entre las causas se pueden considerar la falta de tiempo asignado al taller,
% las limitaciones en cuanto a laboratorios, el poco personal, la falta de una
% correcta selección de los contenidos, así como, una calificación que asegure el
% cumplimiento de los objetivos.

\section{Resume}