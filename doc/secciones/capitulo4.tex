\chapter{ EXPERIMENTACIÓN Y RESULTADOS PRELIMINARES}

% \section{Protocolo de entrenamiento}



\section{Experimentos}

\begin{table}[h!]
\centering
\caption{Evaluation Metrics: Mean, Std, and Median per Model}
\resizebox{\textwidth}{!}{%
\begin{tabular}{l|cc|cc}
\toprule
\multirow{\textbf{Metric}} 
& \multicolumn{2}{c|}{\textbf{GPT-4o}} 
& \multicolumn{2}{c}{\textbf{Deepseek-chat}} \\
& \textbf{Mean} & \textbf{Std}  
& \textbf{Mean} & \textbf{Std}  \\
\midrule
pass@k        & 0.0 & 0.0  & 0.0 & 0.0   \\
maj@k         & 0.0 & 0.0  & 0.0 & 0.0   \\
code\_bleu@k  & 0.273 & 0.045  & 0.304 & 0.042   \\
perc\_error@k & 1.0 & 0.0  & 1.0 & 0.0   \\
\bottomrule
\end{tabular}
}
\label{tab:metrics_models}
\end{table}


\begin{table}[h!]
\centering
\caption{Evaluation Metrics of Qwen2.5-code-7b: Mean, Std, and Median per Model Checkpoint}
\resizebox{\textwidth}{!}{%
\begin{tabular}{l|cc|cc}
\toprule
\textbf{Metric} 
& \multicolumn{2}{c|}{\textbf{Checkpoint 20}} 
& \multicolumn{2}{c}{\textbf{Checkpoint 320}} \\
& \textbf{Mean} & \textbf{Std} 
& \textbf{Mean} & \textbf{Std}  \\
\midrule
pass@k &  0.115 & 0.319  & 0.12 & 0.324   \\
maj@k         & 0.046 & 0.209  & 0.054 & 0.227    \\
code\_bleu@k  & 0.36 & 0.065  & 0.357 & 0.227   \\
perc\_error@k & 0.402 & 0.213  & 0.419 & 0.219    \\
\bottomrule
\end{tabular}
}
\label{tab:metrics_checkpoints}
\end{table}


\begin{table}[h!]
\centering
\caption{Evaluation Metrics of Llama3: Mean, Std, and Median }
\resizebox{\textwidth}{!}{%
\begin{tabular}{l|ccc}
\toprule
\textbf{Metric} & \textbf{Mean} & \textbf{Std} & \textbf{Median} \\
\midrule
pass@k        & 0.410 & 0.492 & 0.0 \\
maj@k         & 0.230 & 0.421 & 0.0 \\
code\_bleu@k  & - & - & - \\
perc\_error@k & 0.418 & 0.248 & 0.375 \\
\bottomrule
\end{tabular}
}
\label{tab:evaluation_metrics}
\end{table}

\begin{table}[h!]
\centering
\caption{Evaluation Metrics: Mean, Std, and Median}
\resizebox{\textwidth}{!}{%
\begin{tabular}{l|ccc}
\toprule
\textbf{Metric} & \textbf{Mean} & \textbf{Std} & \textbf{Median} \\
\midrule
perc\_error@ & 0.233 & 0.299 & 0.125 \\
pass@k    & 0.742 & 0.437 & 1.0   \\
pass\_over\_k  & 0.460 & 0.410 & 0.406 \\
code\_bleu@k  & 0.355 & 0.044 & 0.353 \\
\bottomrule
\end{tabular}
}
\label{tab:metrics_generated}
\end{table}

% \subsection{Protocolo de experimentación}

\section{Resultados y discusión}

% \chapter{MARCO METODOLÓGICO}

% Inicie aquí el texto, utilizando sangría de 1.25 cm. en el primer
% párrafo. Continúe el segundo párrafo.

% Continúe el segundo párrafo

% \section{Primer subtítulo}

% En la actualidad, las unidades de información hacen frente a muchos cambios
% debido a los avances tecnológicos, explosión informativa, nuevos recursos y
% soportes, por lo cual se implementan servicios innovadores que les permitan a
% sus usuarios tener acceso a muchas fuentes de información. Para asegurar el
% acceso y uso de los servicios los usuarios requieren poseer una serie de
% habilidades que les permitan identificar, recuperar, manejar, discernir,
% organizar, utilizar y comunicar la información de manera eficaz para la toma de
% decisiones.

% Entre las causas se pueden considerar la falta de tiempo asignado al taller,
% las limitaciones en cuanto a laboratorios, el poco personal, la falta de una
% correcta selección de los contenidos, así como, una calificación que asegure el
% cumplimiento de los objetivos.

