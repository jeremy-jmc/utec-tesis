\chapter{ EXPERIMENTS AND RESULTS}

This chapter presents the outcomes derived from applying the methodology detailed in the preceding chapter. We begin by describing the experimental setup and specifying the hyperparameters employed. Next, we report the results from the evaluated models, covering both quantitative and qualitative analyses. Lastly, we discuss the significance of these findings and include case studies to demonstrate the practical relevance of our results.

\section{Experimental setup}

\section{Hyperparameter selection}

\section{Quantitative Results}

% TODO: TABLE 1 comparing Phi3-mini4k-instruct, Llama3-8B-Instruct, baseline and trained models along with Closed-Source models like GPT-4o-models, pass@k with k = 1 (with greedy decoding), 4, 16 (multinomial sampling)

\begin{table}[h!]
\centering
\caption{Evaluation Metrics finetuned LLaMA3: Mean, Std, and Median (Test Set)}
% \resizebox{\textwidth}{!}
{%
\begin{tabular}{l|ccc}
\toprule
\textbf{Metric} & \textbf{Mean} & \textbf{Std} & \textbf{Median} \\
\midrule
perc\_error@ & 0.233 & 0.299 & 0.125 \\
pass@k    & 0.742 & 0.437 & 1.0   \\
pass\_over\_k  & 0.460 & 0.410 & 0.406 \\
code\_bleu@k  & 0.355 & 0.044 & 0.353 \\
\bottomrule
\end{tabular}
}
\label{tab:metrics_generated}
\end{table}

\begin{table}[h!]
\centering
\caption{Evaluation Metrics for o4-mini: Mean, Std, and Median (Test Set)}
{%
\begin{tabular}{l|ccc}
\toprule
\textbf{Metric} & \textbf{Mean} & \textbf{Std} & \textbf{Median} \\
\midrule
perc\_error@k     & 0.003 & 0.014 & 0.0   \\
pass\_at@k        & 0.303 & 0.459 & 0.0   \\
pass\_over\_k      & 0.063 & 0.175 & 0.0   \\
code\_bleu@k      & 0.370 & 0.054 & 0.375 \\
\bottomrule
\end{tabular}
}
\label{tab:metrics_new_data}
\end{table}


\section{Discussion}

Based on the observed metrics, several conclusions emerge regarding the performance of the fine-tuned LLaMA3‑8B‑Instruct model versus the base o4‑mini. First, LLaMA3‑8B‑Instruct exhibits a higher perc\_error@k than o4‑mini; yet it significantly outperforms o4‑mini in both pass@k and pass\_over\_k. This indicates that, despite a slightly lower rate of successful compilation, the fine‑tuned model produces more precise and semantically relevant solutions. In contrast, o4‑mini’s better perc\_error@k can be misleading: it often compiles code that runs but fails to solve the problem correctly.

This discrepancy is likely rooted in architectural and training differences. o4‑mini is believed to use a Mixture‑of‑Experts (MoE) setup with approximately 40B total parameters—only about 8B of which are active per inference—whereas LLaMA3‑8B‑Instruct is a dense model with all 8B parameters actively contributing . While MoE architectures provide access to a broader parameter pool, this capacity advantage does not automatically translate to superior domain-specific performance. In contrast, the fully fine‑tuned LLaMA model has internalized domain-specific patterns—such as those needed to manipulate crime data (locations, dates, crime types, etc.)—enabling it to generate more accurate and relevant code.
\section{Qualitative Analysis}

% \section{Discussion}

\section{Case studies}

\subsection{Case I: Counterfactual Evaluation of Targeted Interventions in High-Crime Zones of Suzhou (Jiangsu)}

\begin{itemize}
    \item \textbf{Case summary:} This analysis examines the hypothetical effects of intensive police interventions targeting the most crime-prone streets in Suzhou, Jiangsu Province. The study is motivated by the growing interest in situational crime prevention strategies \cite{clarke1995situational}, which propose that altering or enhancing surveillance in urban areas can influence criminal behavior. 
        \item \textbf{General Objective:} To evaluate the potential impact of securing the five most dangerous streets in Suzhou on the city’s overall crime rates, using data collected between 2020 and 2022.
        \item \textbf{Justification:} The counterfactual methodology provides insights into a critical public policy question: “What would be the outcome if criminal activities in key urban hotspots were entirely prevented?” Leveraging generative AI for this estimation aids in prioritizing security resources and assessing the cost-effectiveness of targeted interventions without requiring real-world implementation. Additionally, Suzhou’s status as a highly developed city in Jiangsu makes it an ideal setting for studying predictive-preventive urban strategies.
        \item \textbf{Proposed questions:}
        \begin{itemize}
            \item "How would securing the top 5 most dangerous streets in Suzhou affect the city’s overall crime rates?"
            \item "What percentage reduction in overall crime could be achieved if all night-time thefts in Suzhou during 2021 were prevented?"
            \item "If theft crime-category decreased by 10\% annually, in which year would Suzhou city record fewer than 500 such crimes?"
            \item "If crime reporting increased by 15\% in Suzhou city, how would that affect the city's safety ranking in Jiangsu province?"
        \end{itemize}




\end{itemize}


\subsection{Case II: Geospatial Queries}


% \chapter{MARCO METODOLÓGICO}

% Inicie aquí el texto, utilizando sangría de 1.25 cm. en el primer
% párrafo. Continúe el segundo párrafo.

% Continúe el segundo párrafo

% \section{Primer subtítulo}

% En la actualidad, las unidades de información hacen frente a muchos cambios
% debido a los avances tecnológicos, explosión informativa, nuevos recursos y
% soportes, por lo cual se implementan servicios innovadores que les permitan a
% sus usuarios tener acceso a muchas fuentes de información. Para asegurar el
% acceso y uso de los servicios los usuarios requieren poseer una serie de
% habilidades que les permitan identificar, recuperar, manejar, discernir,
% organizar, utilizar y comunicar la información de manera eficaz para la toma de
% decisiones.

% Entre las causas se pueden considerar la falta de tiempo asignado al taller,
% las limitaciones en cuanto a laboratorios, el poco personal, la falta de una
% correcta selección de los contenidos, así como, una calificación que asegure el
% cumplimiento de los objetivos.

