\chapter{ EXPERIMENTS AND RESULTS}

% \section{Protocolo de entrenamiento}
% \subsection{Protocolo de experimentación}

\section{Experiments}

\section{Results}

\begin{table}[h!]
\centering
\caption{Evaluation Metrics: Mean, Std, and Median}
% \resizebox{\textwidth}{!}
{%
\begin{tabular}{l|ccc}
\toprule
\textbf{Metric} & \textbf{Mean} & \textbf{Std} & \textbf{Median} \\
\midrule
perc\_error@ & 0.233 & 0.299 & 0.125 \\
pass@k    & 0.742 & 0.437 & 1.0   \\
pass\_over\_k  & 0.460 & 0.410 & 0.406 \\
code\_bleu@k  & 0.355 & 0.044 & 0.353 \\
\bottomrule
\end{tabular}
}
\label{tab:metrics_generated}
\end{table}

\section{Discussion}

\section{Case studies}

\subsection{Case I:}

\subsection{Case II:}

\subsection{Case III:}

% \chapter{MARCO METODOLÓGICO}

% Inicie aquí el texto, utilizando sangría de 1.25 cm. en el primer
% párrafo. Continúe el segundo párrafo.

% Continúe el segundo párrafo

% \section{Primer subtítulo}

% En la actualidad, las unidades de información hacen frente a muchos cambios
% debido a los avances tecnológicos, explosión informativa, nuevos recursos y
% soportes, por lo cual se implementan servicios innovadores que les permitan a
% sus usuarios tener acceso a muchas fuentes de información. Para asegurar el
% acceso y uso de los servicios los usuarios requieren poseer una serie de
% habilidades que les permitan identificar, recuperar, manejar, discernir,
% organizar, utilizar y comunicar la información de manera eficaz para la toma de
% decisiones.

% Entre las causas se pueden considerar la falta de tiempo asignado al taller,
% las limitaciones en cuanto a laboratorios, el poco personal, la falta de una
% correcta selección de los contenidos, así como, una calificación que asegure el
% cumplimiento de los objetivos.

