\chapter*{\center \Large APPENDIX} 
\addcontentsline{toc}{section}{\bfseries APPENDIX} 
\markboth{APPENDIX}{APPENDIX} 

% \par Los algoritmos desarrollados .....
\appendix
\renewcommand{\thesection}{A.\arabic{section}}


\section{Dataset specification}
\label{appendix:dataset}



\begin{table}[H]
\centering
\caption{Dataset Description: \texttt{crimes\_df} – Crime Records}
\begin{tabular}{|l|l|p{8cm}|}
\hline
\textbf{Column} & \textbf{Type} & \textbf{Description} \\
\hline
case\_type & string & Category of the crime incident (e.g., theft, assault). \\
latitude & float & Latitude of the incident location. \\
longitude & float & Longitude of the incident location. \\
incident\_location & string & Textual description of the incident location. \\
incident\_province & string & Province where the incident occurred. \\
incident\_city & string & City where the incident occurred. \\
incident\_county & string & County or district of the incident. \\
formatted\_datetime & datetime & Standardized timestamp of the incident. \\
street\_name & string & Name of the street where the incident occurred. \\
geometry & geospatial point & Geographic point of the incident. \\
\hline
\end{tabular}
\end{table}


\begin{table}[H]
\centering
\caption{Dataset Description: \texttt{streets\_df} – Street Network}
\begin{tabular}{|l|l|p{8cm}|}
\hline
\textbf{Column} & \textbf{Type} & \textbf{Description} \\
\hline
street\_name & string & Official name of the street (can be linked to \texttt{crimes\_df.street\_name}). \\
geometry & line polygon & Geometric representation of the street. \\
incident\_province & string & Province where the street is located. \\
\hline
\end{tabular}
\end{table}


\begin{table}[H]
\centering
\caption{Dataset Description: \texttt{geometries\_df} – Administrative Boundaries}
\begin{tabular}{|l|l|p{8cm}|}
\hline
\textbf{Column} & \textbf{Type} & \textbf{Description} \\
\hline
name & string & Name of the administrative region. \\
geometry & polygon & Boundary geometry of the region. \\
geom\_type & string & Administrative level (e.g., province, city, county). \\
\hline
\end{tabular}
\end{table}


% Later in your document, you can reference this appendix with:
% As shown in Appendix~\ref{appendix:dataset}, the dataset...

\section{Prompts}
\label{appendix:prompts}


\begin{lstlisting}[language=txt, caption={Prompt for generating Python code solutions}, label={lst:prompt_code_generation}, basicstyle=\ttfamily\small, breaklines=true, columns=fullflexible, keepspaces=true]

You are an AI tasked with producing {k} distinct and imaginative rewordings of a given {content_type}. Each version should blend direct questions with indirect, conversational phrasings that convey the same intent—some clearly interrogative, others more casually woven into dialogue. Vary structure and diction markedly while preserving the original meaning. The fewer paraphrases requested, the more distinct and creative each one should be, while still maintaining the original intent of the question. Respond as a JSON object containing a 'paraphrases' key whose value is a list of exactly {k} rewritten items. Re‑order arguments when necessary, since the prompts are template‑generated. Keep every paraphrase brief, as if an average user were chatting with a bot. Any terms wrapped in <> must remain unchanged. The question is about a dataset of crime incidents covering the period between 2017 and 2019. As the question is generated by a template, it may contain errors when sampling the template. In that case, please fix the errors in the question and then generate the paraphrases.
\end{lstlisting}